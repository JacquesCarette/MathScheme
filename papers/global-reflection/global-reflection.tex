\documentclass[fleqn]{article}

\usepackage{latexsym}
\usepackage{amssymb}
\usepackage{amsmath}
\usepackage{stmaryrd}
\usepackage{phonetic}
\usepackage{graphicx}
\usepackage{xcolor}
\usepackage{xurl}
\usepackage{hyperref}
\hypersetup{
  colorlinks = true,
  linkcolor={blue!70!black},
  citecolor={blue!70!black},
  urlcolor={blue!70!black}
}
\usepackage{listings}
\lstset{language=ml,basicstyle=\ttfamily\small,breaklines=true,showspaces=false,
  showstringspaces=false,breakatwhitespace=true,texcl=true,
  escapeinside={(*}{*)}}

%% Requires:
%% 
%% \usepackage{latexsym}
%% \usepackage{amssymb}
%% \usepackage{stmaryrd}

\renewcommand{\labelenumi}{(\theenumi)}
%\renewcommand{\labelenumi}{\theenumi.}
%\renewcommand{\labelenumii}{\theenumii.}
%\renewcommand{\labelenumiii}{\theenumiii.}
\newcommand{\be}{\begin{enumerate}}
\newcommand{\ee}{\end{enumerate}}
\newcommand{\bi}{\begin{itemize}}
\newcommand{\ei}{\end{itemize}}
\newcommand{\bc}{\begin{center}}
\newcommand{\ec}{\end{center}}
\newcommand{\bsp}{\begin{sloppypar}}
\newcommand{\esp}{\end{sloppypar}}

\newcommand{\sglsp}{\ }
\newcommand{\dblsp}{\ \ }
\newcommand{\bs}{\mbox{\textvisiblespace}}

\newcommand{\sA}{\mbox{$\cal A$}}
\newcommand{\sB}{\mbox{$\cal B$}}
\newcommand{\sC}{\mbox{$\cal C$}}
\newcommand{\sD}{\mbox{$\cal D$}}
\newcommand{\sE}{\mbox{$\cal E$}}
\newcommand{\sF}{\mbox{$\cal F$}}
\newcommand{\sG}{\mbox{$\cal G$}}
\newcommand{\sH}{\mbox{$\cal H$}}
\newcommand{\sI}{\mbox{$\cal I$}}
\newcommand{\sJ}{\mbox{$\cal J$}}
\newcommand{\sK}{\mbox{$\cal K$}}
\newcommand{\sL}{\mbox{$\cal L$}}
\newcommand{\sM}{\mbox{$\cal M$}}
\newcommand{\sN}{\mbox{$\cal N$}}
\newcommand{\sO}{\mbox{$\cal O$}}
\newcommand{\sP}{\mbox{$\cal P$}}
\newcommand{\sQ}{\mbox{$\cal Q$}}
\newcommand{\sR}{\mbox{$\cal R$}}
\newcommand{\sS}{\mbox{$\cal S$}}
\newcommand{\sT}{\mbox{$\cal T$}}
\newcommand{\sU}{\mbox{$\cal U$}}
\newcommand{\sV}{\mbox{$\cal V$}}
\newcommand{\sW}{\mbox{$\cal W$}}
\newcommand{\sX}{\mbox{$\cal X$}}
\newcommand{\sY}{\mbox{$\cal Y$}}
\newcommand{\sZ}{\mbox{$\cal Z$}}

\newcommand{\bA}{\mathbb{A}}
\newcommand{\bB}{\mathbb{B}}
\newcommand{\bC}{\mathbb{C}}
\newcommand{\bD}{\mathbb{D}}
\newcommand{\bE}{\mathbb{E}}
\newcommand{\bF}{\mathbb{F}}
\newcommand{\bG}{\mathbb{G}}
\newcommand{\bH}{\mathbb{H}}
\newcommand{\bI}{\mathbb{I}}
\newcommand{\bJ}{\mathbb{J}}
\newcommand{\bK}{\mathbb{K}}
\newcommand{\bL}{\mathbb{L}}
\newcommand{\bM}{\mathbb{M}}
\newcommand{\bN}{\mathbb{N}}
\newcommand{\bO}{\mathbb{O}}
\newcommand{\bP}{\mathbb{P}}
\newcommand{\bQ}{\mathbb{Q}}
\newcommand{\bR}{\mathbb{R}}
\newcommand{\bS}{\mathbb{S}}
\newcommand{\bT}{\mathbb{T}}
\newcommand{\bU}{\mathbb{U}}
\newcommand{\bV}{\mathbb{V}}
\newcommand{\bW}{\mathbb{W}}
\newcommand{\bX}{\mathbb{X}}
\newcommand{\bY}{\mathbb{Y}}
\newcommand{\bZ}{\mathbb{Z}}

\renewcommand{\phi}{\varphi}
\newcommand{\seq}[1]{{\langle #1 \rangle}}
\newcommand{\set}[1]{{\{ #1 \}}}
\newcommand{\tuple}[1]{{( #1 )}}
\newcommand{\mlist}[1]{{[ #1 ]}}
\newcommand{\sembrack}[1]{\llbracket#1\rrbracket}
\newcommand{\synbrack}[1]{\ulcorner#1\urcorner}
\newcommand{\commabrack}[1]{\lfloor#1\rfloor}
\newcommand{\bsynbrack}[1]{\lceil#1\rceil}
\newcommand{\bsembrack}[1]{\lceil\!\!\lceil#1\rceil\!\!\rceil}
\newcommand{\mname}[1]{\mbox{\sf #1}}
\newcommand{\mcolon}{\mathrel:}
\newcommand{\mdot}{\mathrel.}
\newcommand{\bo}{\omicron}
\newcommand{\modpar}{\models_{\rm par}}
\newcommand{\modreg}{\models_{\rm reg}}
\newcommand{\proves}[2]{#1 \vdash #2}
\newcommand{\notproves}[2]{#1 \not\vdash #2}
\newcommand{\provesin}[3]{#1 \vdash_{#2} #3}
\newcommand{\notprovesin}[3]{#1 \not\vdash_{#2} #3}
%\newcommand{\leqq}[1]{\mathrel{\preceq_{#1}}}
\newcommand{\parrow}{\rightharpoonup}
\newcommand{\tarrow}{\rightarrow}
\newcommand{\term}{\seq}
\newcommand{\lub}{\sqcup}
\newcommand{\subfun}{\sqsubseteq}
\newcommand{\strictsubfun}{\sqsubset}
\newcommand{\subpred}{\subseteq}
\newcommand{\strictsubpred}{\subset}
\newcommand{\BoxApp}{\Box\,}
\newcommand{\BOX}{\mathrel{\Box}}
\newcommand{\funapp}{\mathrel@}
\newcommand{\StarApp}{\star\,}
\newcommand{\compl}[1]{\overline{#1}}
\newcommand{\recip}[1]{{#1}^{-1}}

\newcommand{\com}{\mname{complement}}
\newcommand{\dom}{\mname{domain}}
\newcommand{\sumcl}{\mname{sum}}
\newcommand{\pow}{\mname{power}}
\newcommand{\pair}{\mname{pair}}
\newcommand{\opair}{\mname{ordered-pair}}
\newcommand{\inters}{\mname{intersection}}
\newcommand{\emp}{\mname{empty}}
\newcommand{\uni}{\mname{univocal}}
\newcommand{\fun}{\mname{function}}
\newcommand{\card}{\mname{card}}
\newcommand{\sets}{\mname{sets}}
\newcommand{\monotone}{\mname{monotone}}
\newcommand{\continuous}{\mname{continuous}}
\newcommand{\chain}{\mname{chain}}
\newcommand{\mub}{\mname{ub}}
\newcommand{\mlub}{\mname{lub}}
\newcommand{\fixedpoint}{\mname{fp}}
\newcommand{\leastfixedpoint}{\mname{lfp}}
\newcommand{\strongfixedpoint}{\mname{sfp}}
\newcommand{\emptyfun}{\triangle}
\newcommand{\emptylist}{[\,]}
\newcommand{\consext}{\trianglelefteq}
\newcommand{\der}[2]{\stackrel[#2]{#1}{\longrightarrow}}

\newcommand{\Iota}{\mbox{\rm I}}
\newcommand{\IotaApp}{\mbox{\rm I}\,}
\newcommand{\iotaApp}{\iota\,}
\newcommand{\invertediota}{\rotatebox[origin=c]{180}{$\iota$}}
\newcommand{\Epsilon}{\mbox{\rm E}}
\newcommand{\EpsilonApp}{\mbox{\rm E}\,}
\newcommand{\epsilonApp}{\epsilon\,}
\newcommand{\True}{\mbox{\sf T}} 
\newcommand{\False}{\mbox{\sf F}} 
\newcommand{\Trueword}{\sf true}
\newcommand{\Falseword}{\sf false}
\newcommand{\Neg}{\neg} 

\ifdefined \And 
  \renewcommand{\And}{\wedge}
\else
  \newcommand{\And}{\wedge}
\fi

\newcommand{\AND}{\mathrel\&}
\newcommand{\Or}{\vee}
\newcommand{\Implies}{\supset}
\newcommand{\ImpliesAlt}{\Rightarrow}
\newcommand{\ImpliesTrad}{\rightarrow}
\newcommand{\Iff}{\equiv}
\newcommand{\IffAlt}{\Leftrightarrow}
\newcommand{\IffTrad}{\leftrightarrow}
\newcommand{\Sheffer}{\mathrel|}
\newcommand{\ShefferAlt}{\mathrel\uparrow}
\newcommand{\PeirceAlt}{\mathrel\downarrow}
\newcommand{\ExOrAlt}{\oplus}
\newcommand{\Forall}{\forall}
\newcommand{\ForallApp}{\forall\,}
\newcommand{\Forsome}{\exists}
\newcommand{\ForsomeApp}{\exists\,}
\newcommand{\ForsomeUniqueApp}{\exists!\,}
\newcommand{\IsDef}{\downarrow}
\newcommand{\IsUndef}{\uparrow}
\newcommand{\Equal}{=}
\newcommand{\QuasiEqual}{\simeq}
\newcommand{\Undefined}{\bot}
\newcommand{\If}{\mname{if}}
\newcommand{\IfAlt}[3]{#1 \mapsto #2|#3}
\newcommand{\IsDefApp}{\!\IsDef}
\newcommand{\IsUndefApp}{\!\IsUndef}
\newcommand{\TRUE}{\mbox{{\sc t}}}
\newcommand{\FALSE}{\mbox{{\sc f}}}
\newcommand{\truthvalues}{\{\TRUE,\FALSE\}}
\newcommand{\LambdaApp}{\lambda\,}
\newcommand{\LAMBDAapp}{\Lambda\,}
\newcommand{\PiApp}{\Pi\,}
\newcommand{\AlphaEquiv}{\stackrel{\alpha}{=}}
\newcommand{\SigmaApp}{\Sigma\,}
\newcommand{\SetAbs}{\mbox{\rm S}}
\newcommand{\SetAbsApp}{\mbox{\rm S}\,}
\newcommand{\ClassAbs}{\mbox{\sc C}}
\newcommand{\ClassAbsApp}{\mbox{\sc C}\,}
\newcommand{\DepTypeProd}{\otimes}
\newcommand{\DepTypeProdApp}{\otimes\,}

\newcommand{\mterm}[2]{\textbf{term}_{#1}[#2]}
\newcommand{\mterma}[3]{\textbf{term}_{#1}[#2,#3]}
\newcommand{\mform}[2]{\textbf{form}_{#1}[#2]}
\newcommand{\mforma}[3]{\textbf{form}_{#1}[#2,#3]}
\newcommand{\mkind}[2]{\textbf{kind}_{#1}[#2]}
\newcommand{\mtype}[2]{\textbf{type}_{#1}[#2]}
\newcommand{\mtypea}[3]{\textbf{type}_{#1}[#2,#3]}
\newcommand{\mvar}[3]{\textbf{var}_{#1}[#2,#3]}
\newcommand{\mexpr}[3]{\textbf{expr}_{#1}[#2,#3]}
\newcommand{\mexpra}[2]{\textbf{expr}_{#1}[#2]}
\newcommand{\mins}[3]{\textbf{ins}_{#1}[#2,#3]}

\iffalse
\newcommand{\mvar}[3]{\textbf{var}_{#1}(#2,#3)}
\newcommand{\mterm}[3]{\textbf{term}_{#1}(#2,#3)}
\newcommand{\mform}[2]{\textbf{form}_{#1}(#2)}
\newcommand{\mtype}[2]{\textbf{type}_{#1}(#2)}
\newcommand{\mexpr}[3]{\textbf{expr}_{#1}(#2,#3)}
\fi

\newcommand{\imps}{\mbox{\sc imps}}
\newcommand{\fol}{\mbox{\sc fol}}
\newcommand{\msfol}{\mbox{\sc ms-fol}}
\newcommand{\lutins}{\mbox{\sc lutins}}
\newcommand{\bestt}{\mbox{\sc bestt}}
\newcommand{\stt}{\mbox{\sc stt}}
\newcommand{\sttwu}{\mbox{{\sc stt}{\small w}{\sc u}}}
\newcommand{\sttws}{\mbox{{\sc stt}{\small w}{\sc s}}}
\newcommand{\vlisp}{\mbox{\sc vlisp}}
\newcommand{\vmach}{\mbox{\sc vmach}}
\newcommand{\gnu}{\mbox{\sc gnu}}
\newcommand{\ml}{\mbox{\sc ml}}
\newcommand{\zf}{\mbox{\sc zf}}
\newcommand{\nbg}{\mbox{\sc nbg}}
\newcommand{\pnbg}{\mbox{\sc pnbg}}
\newcommand{\snbg}{\mbox{\sc snbg}}
\newcommand{\pfol}{\mbox{\sc pfol}}
\newcommand{\nbgstar}{$\mbox{\sc nbg}^\ast$}
\newcommand{\boldnbgstar}{$\mbox{\bf NBG}^\ast$}
\newcommand{\pf}{${\bf PF}$}
\newcommand{\pfstar}{${\bf PF}^\ast$}
\newcommand{\churchqe}{$\mbox{\sc ctt}_{\rm qe}$}
\newcommand{\churchuqe}{$\mbox{\sc ctt}_{\rm uqe}$}
\newcommand{\qzero}{${\cal Q}_0$}
\newcommand{\qzerou}{${\cal Q}^{\rm u}_{0}$}
\newcommand{\qzerouqe}{${\cal Q}^{\rm uqe}_{0}$}

\newcommand{\eves}{\mbox{\sc eves}}
\newcommand{\hol}{\mbox{\sc hol}}
\newcommand{\mizar}{Mizar}
\newcommand{\nqthm}{Nqthm}
\newcommand{\pvs}{\mbox{\sc pvs}}
\newcommand{\stmm}{\mbox{\sc stmm}}
\newcommand{\stmmstar}{$\mbox{\sc stmm}^\ast$}
\newcommand{\ehdm}{\mbox{\sc ehdm}}
\newcommand{\obj}{\mbox{\sc obj}3}
\newcommand{\iotasys}{\mbox{\sc iota}}
\newcommand{\kids}{\mbox{\sc kids}}
\newcommand{\cstt}{\mbox{\sc cstt}}
\newcommand{\tps}{\mbox{\sc tps}}
\newcommand{\lego}{\mbox{\sc lego}}
\newcommand{\Blang}{\mbox{\sc b}}
\newcommand{\zed}{\mbox{\sc z}}
\newcommand{\vdm}{\mbox{\sc vdm}}
\newcommand{\vdmsl}{\mbox{\sc vdm-sl}}
\newcommand{\bee}{\mbox{\sc b}}

\newtheorem{theorem}{Theorem}
\newtheorem{corollary}{Corollary}
\newtheorem{lemma}{Lemma}
\newtheorem{proposition}{Proposition}
\newtheorem{remark}{Remark}
\newtheorem{example}{Example}
\newtheorem{definition}{Definition}
\newtheorem{axschemas}{Axiom Schemas}
\newtheorem{infrule}{Rule}
\newtheorem{infrules}{Rules}
\newtheorem{requirement}{Requirement}
\newtheorem{antirequirement}{Antirequirement}
\newtheorem{designgoal}{Design Goal}

\newtheorem{thm}{Theorem}[section]
\newtheorem{cor}[thm]{Corollary}
\newtheorem{lem}[thm]{Lemma}
\newtheorem{prop}[thm]{Proposition}
\newtheorem{rem}[thm]{Remark}
\newtheorem{eg}[thm]{Example}
\newtheorem{df}[thm]{Definition}
\newtheorem{conj}[thm]{Conjecture}

\newenvironment{proof}{\par\noindent{\bf Proof\sglsp}}{\hfill$\Box$}
\newenvironment{proofsketch}{\par\noindent{\bf Proof sketch\sglsp}}{\hfill$\Box$}

\newenvironment{namedform}[1]
   {\begin{tabbing}\textbf{#1}\ } {\end{tabbing}}

\newcommand{\urlpart}[1]{\mbox{\texttt{#1}}\linebreak[0]}

\newcommand{\pnote}[1]{{\langle \text{#1} \rangle}}


\newcommand{\HOL}{$\mbox{\rm HOL}$}
\newcommand{\HL}{$\mbox{\rm HOL Light}$}
\newcommand{\HOLtwoP}{$\mbox{\rm HOL2P}$}
\newcommand{\HLQE}{$\mbox{\rm HOL Light QE}$}


\title{{\bf Global Reflection}}

\author{Jacques Carette \and William M. Farmer \and Johanna Schwartzentruber}

\date{today}

\begin{document}

\maketitle

\begin{abstract}

\end{abstract}

\section{Introduction}\label{sec:introduction}

\section{Global Reflection}\label{sec:global-reflection}

\section{${\rm {\bf CTT_{qe}}}$}\label{sec:ctt_qe}

The syntax, semantics, and proof system of {\churchqe}
are defined in~\cite{Farmer18}.  Here
we will only introduce the definitions and results of that
are key to understanding how \HLQE{} implements {\churchqe}.  The
reader is encouraged to consult~\cite{Farmer18} when additional
details are required.

\subsection{Syntax}

{\churchqe} has the same machinery as {\qzero} plus an
inductive type $\epsilon$ of syntactic values, a partial quotation
operator, and a typed evaluation operator.

A \emph{type} of {\churchqe} is defined inductively by the following
formation rules:
%
%\vspace*{-1.5mm}
\be

  \item \emph{Type of individuals}: $\iota$ is a type.

  \item \emph{Type of truth values}: $\omicron$ is a type.

  \item \emph{Type of constructions}: $\epsilon$ is a type.

  \item \emph{Function type}: If $\alpha$ and $\beta$ are types, then
    $(\alpha \tarrow \beta)$ is a type.

\ee

\noindent
Let $\sT$ denote the set of types of {\churchqe}.  
A \emph{typed symbol} is a symbol with a subscript from $\sT$.  Let
$\sV$ be a set of typed symbols such that, for each $\alpha \in \sT$,
$\sV$ contains denumerably many typed symbols with subscript~$\alpha$.
A \emph{variable of type $\alpha$} of {\churchqe} is a member of $\sV$
with subscript~$\alpha$.  $\textbf{x}_\alpha, \textbf{y}_\alpha,
\textbf{z}_\alpha, \ldots$ are syntactic variables ranging over
variables of type $\alpha$. Let $\sC$ be a set of typed symbols
disjoint from $\sV$.  A \emph{constant of type $\alpha$} of
{\churchqe} is a member of $\sC$ with subscript~$\alpha$.
$\textbf{c}_\alpha, \textbf{d}_\alpha, \ldots$ are syntactic variables
ranging over constants of type~$\alpha$.  $\sC$ contains a set of
\emph{logical constants} that include $\mname{app}_{\epsilon \tarrow
  \epsilon \tarrow \epsilon}$, $\mname{abs}_{\epsilon \tarrow \epsilon
  \tarrow \epsilon}$, and $\mname{quo}_{\epsilon \tarrow \epsilon}$.

An \emph{expression of type $\alpha$} of {\churchqe} is defined
inductively by the formation rules below.  $\textbf{A}_\alpha,
\textbf{B}_\alpha, \textbf{C}_\alpha, \ldots$ are syntactic variables
ranging over expressions of type $\alpha$.  An expression is
\emph{eval-free} if it is constructed using just the first five
rules.
%
%\vspace*{-3mm}
\be

  \item \emph{Variable}: $\textbf{x}_\alpha$ is an expression of type
    $\alpha$.

  \item \emph{Constant}: $\textbf{c}_\alpha$ is an expression of type
    $\alpha$.

  \item \emph{Function application}: $(\textbf{F}_{\alpha \tarrow
    \beta} \, \textbf{A}_\alpha)$ is an expression of type $\beta$.

  \item \emph{Function abstraction}: $(\LambdaApp \textbf{x}_\alpha
    \mdot \textbf{B}_\beta)$ is an expression of type $\alpha \tarrow
    \beta$.

  \item \emph{Quotation}: $\synbrack{\textbf{A}_\alpha}$ is an
    expression of type $\epsilon$ if $\textbf{A}_\alpha$ is eval-free.

  \item \emph{Evaluation}: $\sembrack{\textbf{A}_\epsilon}_{{\bf
      B}_\beta}$ is an expression of type $\beta$.

\ee 

\noindent
The sole purpose of the second component $\textbf{B}_\beta$ in an
evaluation $\sembrack{\textbf{A}_\epsilon}_{{\bf B}_\beta}$ is to
establish the type of the evaluation; we will thus write
$\sembrack{\textbf{A}_\epsilon}_{{\bf B}_\beta}$ as
$\sembrack{\textbf{A}_\epsilon}_\beta$.

A \emph{construction} of {\churchqe} is an expression of type
$\epsilon$ defined inductively by:

\be

  \item $\synbrack{\textbf{x}_\alpha}$ is a construction.

  \item $\synbrack{\textbf{c}_\alpha}$ is a construction.

  \item If $\textbf{A}_\epsilon$ and $\textbf{B}_\epsilon$ are
    constructions, then $\mname{app}_{\epsilon \tarrow \epsilon
      \tarrow \epsilon} \, \textbf{A}_\epsilon \,
    \textbf{B}_\epsilon$, $\mname{abs}_{\epsilon \tarrow \epsilon
      \tarrow \epsilon} \, \textbf{A}_\epsilon \,
    \textbf{B}_\epsilon$, and $\mname{quo}_{\epsilon \tarrow \epsilon}
    \, \textbf{A}_\epsilon$ are constructions.

\ee

\noindent
The set of constructions is thus an inductive type whose base elements
are quotations of variables and constants, and whose constructors are
$\mname{app}_{\epsilon \tarrow \epsilon \tarrow \epsilon}$,
$\mname{abs}_{\epsilon \tarrow \epsilon \tarrow \epsilon}$, and
$\mname{quo}_{\epsilon \tarrow \epsilon}$.  As we will see shortly,
constructions serve as syntactic values.

Let $\sE$ be the function mapping eval-free expressions to
constructions that is defined inductively as follows:

\be

  \item $\sE(\textbf{x}_\alpha) = \synbrack{\textbf{x}_\alpha}$.

  \item $\sE(\textbf{c}_\alpha) = \synbrack{\textbf{c}_\alpha}$.

  \item $\sE(\textbf{F}_{\alpha \tarrow \beta} \, \textbf{A}_\alpha) =
    \mname{app}_{\epsilon \tarrow \epsilon \tarrow \epsilon} \,
    \sE(\textbf{F}_{\alpha \tarrow \beta}) \, \sE(\textbf{A}_\alpha)$.

  \item $\sE(\LambdaApp \textbf{x}_\alpha \mdot \textbf{B}_\beta) =
    \mname{abs}_{\epsilon \tarrow \epsilon \tarrow \epsilon} \,
    \sE(\textbf{x}_\alpha) \, \sE(\textbf{B}_\beta)$.

  \item $\sE(\synbrack{\textbf{A}_\alpha}) = \mname{quo}_{\epsilon
    \tarrow \epsilon} \, \sE(\textbf{A}_\alpha)$.

\ee

\noindent
When $\textbf{A}_\alpha$ is eval-free, $\sE(\textbf{A}_\alpha)$ is the
unique construction that represents the syntax tree of
$\textbf{A}_\alpha$.  That is, $\sE(\textbf{A}_\alpha)$ is a syntactic
value that represents how $\textbf{A}_\alpha$ is syntactically
constructed.  For every eval-free expression, there is a construction
that represents its syntax tree, but not every construction represents
the syntax tree of an eval-free expression.  For example,
$\mname{app}_{\epsilon \tarrow \epsilon \tarrow \epsilon} \,
\synbrack{\textbf{x}_\alpha} \, \synbrack{\textbf{x}_\alpha}$
represents the syntax tree of $(\textbf{x}_\alpha \,
\textbf{x}_\alpha)$ which is not an expression of {\churchqe} since
the types are mismatched.  A construction is \emph{proper} if it is in
the range of $\sE$, i.e., it represents the syntax tree of an
eval-free expression.

The purpose of $\sE$ is to define the semantics of quotation: the
meaning of $\synbrack{\textbf{A}_\alpha}$ is $\sE(\textbf{A}_\alpha)$.

\subsection{Semantics}

The semantics of {\churchqe} is based on Henkin-style general
models~\cite{Henkin50}.  An expression $\textbf{A}_\epsilon$ of type
$\epsilon$ denotes a construction, and when $\textbf{A}_\epsilon$ is a
construction, it denotes itself.  The semantics of the quotation and
evaluation operators are defined so that the following two theorems
hold:

\begin{thm}[Law of Quotation] \label{thm:sem-quotation}
$\synbrack{\textbf{A}_\alpha} = \sE(\textbf{A}_\alpha)$ is valid in
  {\churchqe}.
\end{thm}

\begin{cor}$\synbrack{\textbf{A}_\alpha} = \synbrack{\textbf{B}_\alpha}$ is valid in
  {\churchqe} iff $\textbf{A}_\alpha$ and $\textbf{B}_\alpha$ are
  identical expressions.
\end{cor}

\begin{thm}[Law of Disquotation] \label{thm:sem-disquotation}
$\sembrack{\synbrack{\textbf{A}_\alpha}}_\alpha = \textbf{A}_\alpha$
  is valid in {\churchqe}.
\end{thm}

\begin{rem}\em
Notice that this is not the full Law of Disquotation, since only
eval-free expressions can be quoted.  As a result of this restriction,
the liar paradox is not expressible in {\churchqe} and the Evaluation
Problem mentioned above is effectively solved.
\end{rem}

\subsection{Quasiquotation}

Quasiquotation is a parameterized form of quotation in which the
parameters serve as holes in a quotation that are filled with
expressions that denote syntactic values.  It is a very powerful
syntactic device for specifying expressions and defining macros.
Quasiquotation was introduced by Willard Van Orman Quine in 1940 in
the first version of his book \emph{Mathematical
  Logic}~\cite{Quine03}.  It has been extensively employed in the Lisp
family of programming languages~\cite{Bawden99}\footnote{In Lisp, the
  standard symbol for quasiquotation is the backquote ({\tt `})
  symbol, and thus in Lisp, quasiquotation is usually called
  \emph{backquote}.}, and from there to other families of
programming languages, most notably the ML family.

In {\churchqe}, constructing a large quotation from smaller quotations
can be tedious because it requires many applications of the syntax
constructors $\mname{app}_{\epsilon \tarrow \epsilon \tarrow
  \epsilon}$, $\mname{abs}_{\epsilon \tarrow \epsilon \tarrow
  \epsilon}$, and $\mname{quo}_{\epsilon \tarrow \epsilon}$.
Quasiquotation alleviates this problem.
It can be defined straightforwardly in
{\churchqe}.  However, quasiquotation is not part of the official
syntax of {\churchqe}; it is just a notational device used to write
{\churchqe} expressions in a compact form.

As an example, consider $\synbrack{\Neg(\textbf{A}_o \And
  \commabrack{\textbf{B}_\epsilon})}$. Here $\commabrack{\textbf{B}_\epsilon}$
is a \emph{hole} or \emph{antiquotation}. Assume that
$\textbf{A}_o$ contains no holes.  $\synbrack{\Neg(\textbf{A}_o \And
  \commabrack{\textbf{B}_\epsilon})}$ is then an abbreviation for the
verbose expression
\[\mname{app}_{\epsilon \tarrow
  \epsilon \tarrow \epsilon} \, \synbrack{\Neg_{o \tarrow o}} \,
(\mname{app}_{\epsilon \tarrow \epsilon \tarrow \epsilon} \,
(\mname{app}_{\epsilon \tarrow \epsilon \tarrow \epsilon}
\synbrack{\wedge_{o \tarrow o \tarrow o}} \, \synbrack{\textbf{A}_o})
\, \textbf{B}_\epsilon).\] $\synbrack{\Neg(\textbf{A}_o \And
  \commabrack{\textbf{B}_\epsilon})}$ represents the syntax tree
of a negated conjunction in which the part of the tree corresponding
to the second conjunct is replaced by the syntax tree represented by
$\textbf{B}_\epsilon$.  If $\textbf{B}_\epsilon$ is a quotation
$\synbrack{\textbf{C}_o}$, then the quasiquotation
$\synbrack{\Neg(\textbf{A}_o \And
  \commabrack{\synbrack{\textbf{C}_o}})}$ is \emph{equivalent} to the
quotation $\synbrack{\Neg(\textbf{A}_o \And \textbf{C}_o)}$.

\subsection{Proof System}\label{subsec:cttqe-proof-system}

The proof system for {\churchqe} consists of the axioms for {\qzero},
the single rule of inference for {\qzero}, and additional
axioms~\cite[B1--B13]{Farmer18} that define the logical constants
of {\churchqe} (B1--B4, B5, B7), specify $\epsilon$ as an inductive type
(B4, B6), state the properties of quotation and evaluation (B8, B10),
and extend the rules for beta-reduction (B9, B11--13).  We prove
in~\cite{Farmer18} that this proof system is sound for all
formulas and complete for eval-free formulas.

The axioms that express the properties of quotation and evaluation are:

\medskip

\noindent
\begin{minipage}{\textwidth}
  \noindent\textbf{B8 (Properties of Quotation)}
  \be

    \item $\synbrack{\textbf{F}_{\alpha \tarrow \beta} \,
      \textbf{A}_\alpha} = \mname{app}_{\epsilon \tarrow \epsilon
      \tarrow \epsilon} \, \synbrack{\textbf{F}_{\alpha \tarrow
        \beta}} \, \synbrack{\textbf{A}_\alpha}$.

    \item $\synbrack{\LambdaApp \textbf{x}_\alpha \mdot
      \textbf{B}_\beta} = \mname{abs}_{\epsilon \tarrow \epsilon
      \tarrow \epsilon} \, \synbrack{\textbf{x}_\alpha} \,
      \synbrack{\textbf{B}_\beta}$.

    \item $\synbrack{\synbrack{\textbf{A}_\alpha}} =
      \mname{quo}_{\epsilon \tarrow \epsilon} \,
      \synbrack{\textbf{A}_\alpha}$.

  \ee

  \noindent\textbf{B10 (Properties of Evaluation)}

  \be

    \item $\sembrack{\synbrack{\textbf{x}_\alpha}}_\alpha = \textbf{x}_\alpha$.

    \item $\sembrack{\synbrack{\textbf{c}_\alpha}}_\alpha = \textbf{c}_\alpha$.

    \item $(\mname{is-expr}_{\epsilon \tarrow o}^{\alpha \tarrow \beta}
      \, \textbf{A}_\epsilon \And \mname{is-expr}_{\epsilon \tarrow o}^{\alpha}
      \, \textbf{B}_\epsilon) \Implies
          \sembrack{\mname{app}_{\epsilon \tarrow \epsilon
          \tarrow \epsilon} \, \textbf{A}_\epsilon \,
        \textbf{B}_\epsilon}_{\beta} =
      \sembrack{\textbf{A}_\epsilon}_{\alpha \tarrow \beta} \,
      \sembrack{\textbf{B}_\epsilon}_{\alpha}$.

    \item $(\mname{is-expr}_{\epsilon \tarrow o}^{\beta} \,
      \textbf{A}_\epsilon \And \Neg(\mname{is-free-in}_{\epsilon \tarrow
        \epsilon \tarrow o} \, \synbrack{\textbf{x}_\alpha} \,
      \synbrack{\textbf{A}_\epsilon})) \Implies\\[.5ex]
      \hspace*{2ex} \sembrack{\mname{abs}_{\epsilon \tarrow \epsilon
          \tarrow \epsilon} \, \synbrack{\textbf{x}_\alpha} \,
        \textbf{A}_\epsilon}_{\alpha \tarrow \beta} = \LambdaApp
      \textbf{x}_\alpha \mdot \sembrack{\textbf{A}_\epsilon}_\beta$.

    \item $\mname{is-expr}_{\epsilon \tarrow o}^\epsilon \,
      \textbf{A}_\epsilon \Implies \sembrack{\mname{quo}_{\epsilon
          \tarrow \epsilon} \, \textbf{A}_\epsilon}_\epsilon =
      \textbf{A}_\epsilon$.

  \ee

\end{minipage}

\medskip

\noindent
The axioms for extending the rules  for beta-reduction are:

\medskip

\noindent
\begin{minipage}{\textwidth}
  \noindent\textbf{B9 (Beta-Reduction for Quotations)}

  \be

    \item[] $(\LambdaApp \textbf{x}_\alpha \mdot
      \synbrack{\textbf{B}_\beta}) \, \textbf{A}_\alpha =
      \synbrack{\textbf{B}_\beta}$.

  \ee

  \noindent\textbf{B11 (Beta-Reduction for Evaluations)}
  \be

    \item $(\LambdaApp \textbf{x}_\alpha \mdot
      \sembrack{\textbf{B}_\epsilon}_\beta) \, \textbf{x}_\alpha =
      \sembrack{\textbf{B}_\epsilon}_\beta$.
 
    \item $(\mname{is-expr}_{\epsilon \tarrow o}^{\beta} \,
      ((\LambdaApp \textbf{x}_\alpha \mdot \textbf{B}_\epsilon) \,
      \textbf{A}_\alpha) \And %\\[.5ex] \hspace*{0.5ex}
      \Neg(\mname{is-free-in}_{\epsilon \tarrow \epsilon \tarrow o}
      \, \synbrack{\textbf{x}_\alpha} \, ((\LambdaApp
      \textbf{x}_\alpha \mdot \textbf{B}_\epsilon) \,
      \textbf{A}_\alpha))) \Implies \\[.5ex]
      \hspace*{2ex} (\LambdaApp \textbf{x}_\alpha \mdot
      \sembrack{\textbf{B}_\epsilon}_\beta) \, \textbf{A}_\alpha =
      \sembrack{(\LambdaApp \textbf{x}_\alpha \mdot
        \textbf{B}_\epsilon) \, \textbf{A}_\alpha}_\beta$.

  \ee

  \noindent\textbf{B12 (``Not Free In'' means ``Not Effective In'')}

  \bi

    \item[] $\Neg\mname{IS-EFFECTIVE-IN}(\textbf{x}_\alpha,\textbf{B}_\beta)$\\
      where $\textbf{B}_\beta$ is eval-free and
      $\textbf{x}_\alpha$ is not free in $\textbf{B}_\beta$.

  \ei

  \noindent\textbf{B13 (Beta-Reduction for Function Abstractions)}

  \bi

    \item[] $(\Neg \mname{IS-EFFECTIVE-IN}(\textbf{y}_\beta,\textbf{A}_\alpha)\Or 
      \Neg \mname{IS-EFFECTIVE-IN}(\textbf{x}_\alpha,\textbf{B}_\gamma)) \Implies {}\\ 
      \hspace*{2ex}(\LambdaApp \textbf{x}_\alpha \mdot 
      \LambdaApp \textbf{y}_\beta \mdot \textbf{B}_\gamma) \, \textbf{A}_\alpha =
      \LambdaApp \textbf{y}_\beta \mdot 
      ((\LambdaApp \textbf{x}_\alpha \mdot \textbf{B}_\gamma) \, \textbf{A}_\alpha)$\\
      where $\textbf{x}_\alpha$ and $\textbf{y}_\beta$ are distinct.

   \ei

\end{minipage}

\medskip

Substitution is performed using
the properties of beta-reduction as Andrews does in the proof system
for {\qzero}~\cite[p.~213]{Andrews02}.  The following three
beta-reduction cases require discussion:

%\vspace*{-2.5mm}
\be

  \item $(\LambdaApp \textbf{x}_\alpha \mdot \LambdaApp
    \textbf{y}_\beta \mdot \textbf{B}_\gamma) \, \textbf{A}_\alpha$
    where $\textbf{x}_\alpha$ and $\textbf{y}_\beta$ are distinct.

  \item $(\LambdaApp \textbf{x}_\alpha \mdot
      \synbrack{\textbf{B}_\beta}) \, \textbf{A}_\alpha$.

  \item $(\LambdaApp \textbf{x}_\alpha \mdot
    \sembrack{\textbf{B}_\epsilon}_\beta) \, \textbf{A}_\alpha$.

\ee

The first case can normally be reduced when either (1)
$\textbf{y}_\beta$ is not free in $\textbf{A}_\alpha$ or (2)
$\textbf{x}_\alpha$ is not free in $\textbf{B}_\gamma$.  However, due
to the Variable Problem mentioned before, it is only possible to
syntactically check whether a ``variable is not free in an
expression'' when the expression is eval-free.  Our solution
is to replace the syntactic notion of ``a variable is free in
an expression'' by the semantic notion of ``a variable is effective in
an expression'' when the expression is not necessarily eval-free, and
use Axiom B13 to perform the beta-reduction.  

``$\textbf{x}_\alpha$ is effective in $\textbf{B}_\beta$'' means the
value of $\textbf{B}_\beta$ depends on the value of
$\textbf{x}_\alpha$.  Clearly, if $\textbf{B}_\beta$ is eval-free,
``$\textbf{x}_\alpha$ is effective in $\textbf{B}_\beta$'' implies
``$\textbf{x}_\alpha$ is free in $\textbf{B}_\beta$''.  However,
``$\textbf{x}_\alpha$ is effective in $\textbf{B}_\beta$'' is a refinement of
``$\textbf{x}_\alpha$ is free in $\textbf{B}_\beta$'' on eval-free
expressions since $\textbf{x}_\alpha$ is free in $\textbf{x}_\alpha =
\textbf{x}_\alpha$, but $\textbf{x}_\alpha$ is not effective in
$\textbf{x}_\alpha = \textbf{x}_\alpha$.  ``$\textbf{x}_\alpha$ is
effective in $\textbf{B}_\beta$'' is expressed in {\churchqe} as
$\mname{IS-EFFECTIVE-IN}(\textbf{x}_\alpha,\textbf{B}_\beta)$, an
abbreviation for \[\ForsomeApp \textbf{y}_\alpha \mdot ((\LambdaApp
\textbf{x}_\alpha \mdot \textbf{B}_\beta) \, \textbf{y}_\alpha \not=
\textbf{B}_\beta)\] where $\textbf{y}_\alpha$ is any variable of type
$\alpha$ that differs from $\textbf{x}_\alpha$.

The second case is simple since a quotation cannot be modified by
substitution --- it is effectively the same as a constant. Thus
beta-reduction is performed without changing
$\synbrack{\textbf{B}_\beta}$ as shown in Axiom B9 above.

The third case is handled by Axioms B11.1 and B11.2.  B11.1 deals with
the trivial case when $\textbf{A}_\alpha$ is the bound variable
$\textbf{x}_\alpha$ itself.  B11.2 deals with the other much more
complicated situation.  The condition
\[\Neg(\mname{is-free-in}_{\epsilon \tarrow \epsilon \tarrow o}
      \, \synbrack{\textbf{x}_\alpha} \, ((\LambdaApp
      \textbf{x}_\alpha \mdot \textbf{B}_\epsilon) \,
      \textbf{A}_\alpha))\] guarantees that there is no
      \emph{double substitution}.  $\mname{is-free-in}_{\epsilon \tarrow
        \epsilon \tarrow o}$ is a logical constant of {\churchqe} such
      that $\mname{is-free-in}_{\epsilon \tarrow \epsilon \tarrow o}
      \, \synbrack{\textbf{x}_\alpha} \,
      \synbrack{\textbf{B}_\beta}$ says that the variable
      $\textbf{x}_\alpha$ is free in the (eval-free) expression
      $\textbf{B}_\beta$.

Thus we see that substitution in {\churchqe} in the presence of
evaluations may require proving semantic side conditions of the
following two forms:

\be

  \item $\Neg
    \mname{IS-EFFECTIVE-IN}(\textbf{x}_\alpha,\textbf{B}_\beta)$.

  \item $\Neg(\mname{is-free-in}_{\epsilon \tarrow \epsilon \tarrow o}
    \, \synbrack{\textbf{x}_\alpha} \, \synbrack{\textbf{B}_\beta})$.

\ee

\subsection{The Three Design Problems} 

To recap, {\churchqe} solves the three design problems given in
section~\ref{sec:introduction}.  The Evaluation Problem is avoided by
restricting the quotation operator to eval-free expressions and thus
making it impossible to express the liar paradox.  The Variable
Problem is overcome by modifying Andrews' beta-reduction axioms.  The
Double Substitution Problem is eluded by using a beta-reduction axiom
for evaluations that excludes beta-reductions that embody a double
substitution.

\section{HOL Light}\label{sec:hol-light}

\HL~\cite{Harrison09} is an open-source proof assistant developed by
John Harrison.  It implements a logic (HOL) which is a version of
Church's type theory.  It is a simple implementation of the HOL proof
assistant~\cite{GordonMelham93} written in OCaml and hosted on GitHub
at \url{https://github.com/jrh13/hol-light/}.  Although it is a
relatively small system, it has been used to formalize many kinds of
mathematics and to check many proofs including the lion's share of Tom
Hales' proof of the Kepler conjecture~\cite{HalesEtAl17}.

{\HL} is very well suited to serve as a foundation on which to build
an implementation of {\churchqe}: First, it is an open-source system
that can be freely modified as long as certain very minimal conditions
are satisfied.  Second, it is an implementation of a version of simple
type theory that is essentially {\qzero}, the version of Church's type
theory underlying {\churchqe}, plus (1) polymorphic type variables,
(2) an axiom of choice expressed by asserting that the Hilbert
$\epsilon$ operator is a choice (indefinite description) operator, and
(3) an axiom of infinity that asserts that \texttt{ind}, the type of
individuals, is infinite~\cite{Harrison09}.  The type variables in the
implemented logic are not a hindrance; they actually facilitate the
implementation of {\churchqe}.  The presence of the axioms of choice
and infinity in {\HL} alter the semantics of {\churchqe} without
compromising in any way the semantics of quotation and evaluation.
And third, \HL{} supports the definition of inductive types so that
$\epsilon$ can be straightforwardly defined.

\section{HOL Light QE} \label{sec:hol-light-qe}

\subsection{Overview}

{\HLQE} was implemented in four stages:

\be

  \item The set of terms was extended so that {\churchqe}
    expressions could be mapped to {\HL} terms.  This required the
    introduction of \texttt{epsilon}, the type of constructions, and
    term constructors for quotations and evaluations.  See
    subsection~\ref{subsec:mapping}.

  \item The proof system was modified to include the machinery
    in {\churchqe} for reasoning about quotations and evaluations.
    This required adding new rules of inference and modifying
    the \texttt{INST} rule of inference that simultaneously
    substitutes terms $t_1,\ldots,t_n$ for the free variables
    $x_1,\ldots,x_n$ in a sequent.  See
    subsection~\ref{subsec:hlqe-proof-system}.

  \item Machinery --- consisting of {\HOL} function definitions,
    tactics, and theorems --- was created for supporting reasoning
    about quotations and evaluations in the new system.  See
    subsection~\ref{subsec:machinery}.

  \item Examples were developed in the new system to test the
    implementation and to demonstrate the benefits of having quotation
    and evaluation in higher-order logic.  See section~\ref{sec:examples}.

\ee

\noindent
The first and second stages have been completed; both stages involved
modifying the kernel of {\HL}.  The third stage is sufficiently
complete to enable our examples in section~\ref{sec:examples} to work
well and did not involve any further changes to the {\HL} kernel.  We
do expect that adding further examples, which is ongoing, will require
additional machinery but no changes to the kernel.

The {\HLQE} system is being developed at McMaster University and is available at
\[\mbox{\url{https://github.com/JacquesCarette/hol-light-qe}}.\]
Under the supervision of the second and third authors, the core of the
system was developed by the fourth author and the system was later
extended and improved by the first and fifth authors.  The fourth and
fifth authors did their work on undergraduate NSERC USRA research
projects.

\iffalse
It should be further remarked that our fork, from late April 2017, is not fully
up-to-date with respect to {\HL}. In particular, this means that it is best to
compile it with \textsc{OCaml 4.03.0} and \textsc{camlp5 6.16},
both available from \textsc{opam}.
\fi

To run {\HLQE}, execute the following commands in the {\HLQE}
top-level directory named \texttt{hol-light-qe}:

\begin{lstlisting}
1) install opam
2) opam init --comp 4.03.0                                                      
3) opam install "camlp5=6.16" 
4) opam config env
5) cd hol-light-qe
6) make
7) ocaml -I `camlp5 -where` camlp5o.cma                                        
8) #use "hol.ml";;
   #use "Constructions/epsilon.ml";;
   #use "Constructions/pseudoquotation.ml";;
   #use "Constructions/QuotationTactics.ml";;
\end{lstlisting}
\noindent Each test can be run by an appropriate further
\lstinline|#use| statement.

\subsection{Mapping of ${\rm CTT_{qe}}$ Expressions to HOL Terms}\label{subsec:mapping}

\begin{table}[b]
\bc
\begin{tabular}{|lll|}
\hline
\textbf{${\bf CTT_{qe}}$ Type $\alpha$} \hspace*{2ex}
  & \textbf{HOL Type $\mu(\alpha)$}
  & \textbf{Abbreviation for $\mu(\alpha)$}\\
$\omicron$ & \texttt{Tyapp("bool",[])} & \texttt{bool}\\
$\iota$ & \texttt{Tyapp("ind",[])} & \texttt{ind}\\
$\epsilon$ & \texttt{Tyapp("epsilon",[])} & \texttt{epsilon}\\
$\beta \tarrow \gamma$ 
  & \texttt{Tyapp("fun",[\mbox{$\mu(\beta),\mu(\gamma)$}])} \hspace*{2ex}
  & \texttt{\mbox{$\mu(\beta)$}->\mbox{$\mu(\gamma)$}}\\  
\hline
\end{tabular}
\ec
\caption{Mapping of {\churchqe} Types to {\HOL} Types}\label{tab:types} 
\end{table}

\begin{table}[t]
\bc
\begin{tabular}{|ll|}
\hline
\textbf{${\bf CTT_{qe}}$ Expression $e$} \hspace*{2ex}
  & \textbf{HOL Term $\nu(e)$}\\
$\textbf{x}_\alpha$
  & \texttt{Var("x",\mbox{$\mu(\alpha)$})}\\
$\textbf{c}_\alpha$
  & \texttt{Const("c",\mbox{$\mu(\alpha)$})}\\
$\mname{=}_{\alpha \tarrow \alpha \tarrow o}$
  & \texttt{Const("=",\texttt{a\_ty\_var->a\_ty\_var->bool})}\\
$(\textbf{F}_{\alpha \tarrow \beta} \, \textbf{A}_\alpha)$
  & \texttt{Comb(\mbox{\rm $\nu(\textbf{F}_{\alpha \tarrow \beta}),\nu(\textbf{A}_\alpha)$})}\\
$(\LambdaApp \textbf{x}_\alpha \mdot \textbf{B}_\beta)$
  & \texttt{Abs(Var("x",\mbox{$\mu(\alpha)$}),\mbox{\rm $\nu(\textbf{B}_\beta)$})}\\
$\synbrack{\textbf{A}_\alpha}$
  & \texttt{Quote(\mbox{\rm $\nu(\textbf{A}_\alpha),\mu(\alpha)$})}\\
$\sembrack{\textbf{A}_\epsilon}_{{\bf B}_\beta}$
  & \texttt{Eval(\mbox{\rm $\nu(\textbf{A}_\epsilon),\mu(\beta)$})}\\
\hline
\end{tabular}
\ec
\caption{Mapping of {\churchqe} Expressions to {\HOL} Terms}\label{tab:expressions} 
\end{table}

Tables~\ref{tab:types} and~\ref{tab:expressions} illustrate how the
{\churchqe} types and expressions are mapped to the {\HOL} types and
terms, respectively.  The {\HOL} types and terms are written in 
the internal representation employed in {\HLQE}.  The type
\texttt{epsilon} and the term constructors \texttt{Quote} and
\texttt{Eval} are additions to {\HL} explained below.  Since
{\churchqe} does not have type variables, it has a logical constant
$\mname{=}_{\alpha \tarrow \alpha \tarrow o}$ representing equality
for each $\alpha \in \sT$.  The members of this family of constants
are all mapped to a single {\HOL} constant with the polymorphic type
\texttt{a\_ty\_var->a\_ty\_var->bool} where \texttt{a\_ty\_var} is any
chosen {\HOL} type variable.  

The other logical constants of {\churchqe}~\cite[Table 1]{Farmer18}
are not mapped to primitive {\HOL} constants.  $\mname{app}_{\epsilon
  \tarrow \epsilon \tarrow \epsilon}$, $\mname{abs}_{\epsilon \tarrow
  \epsilon \tarrow \epsilon}$, and $\mname{quo}_{\epsilon \tarrow
  \epsilon}$ are implemented by \texttt{App}, \texttt{Abs}, and
\texttt{Quo}, constructors for the inductive type \texttt{epsilon}
given below.  The remaining logical constants are predicates on
constructions that are implemented by {\HOL} functions.
The {\churchqe} type $\epsilon$ is the type of constructions, the
syntactic values that represent the syntax trees of eval-free
expressions.  $\epsilon$ is formalized as an inductive type
\texttt{epsilon}.  Since types are components of terms in
{\HL}, an inductive type \texttt{type} of syntactic values for
{\HLQE} types (which are the same as {\HOL} types) is also
needed.  Specifically:

\begin{lstlisting}
define_type "type = TyVar string
                  | TyBase string
                  | TyMonoCons string type
                  | TyBiCons string type type"

define_type "epsilon = QuoVar string type 
                     | QuoConst string type
                     | App epsilon epsilon
                     | Abs epsilon epsilon
                     | Quo epsilon"
\end{lstlisting}

\noindent
Terms of type \texttt{type} denote the syntax trees of {\HLQE} types,
while the terms of type \texttt{epsilon} denote the syntax trees of
those terms that are eval-free.

The OCaml type of {\HOL} types in {\HLQE}

\begin{lstlisting}
type hol_type = Tyvar of string
              | Tyapp of string * hol_type list
\end{lstlisting}

\noindent
is the same as in {\HL}, but the OCaml type of {\HOL} terms in {\HLQE}

\begin{lstlisting}
type term = Var of string * hol_type
          | Const of string * hol_type
          | Comb of term * term
          | Abs of term * term
          | Quote of term * hol_type
          | Hole of term * hol_type
          | Eval of term * hol_type
\end{lstlisting}

\noindent
has three new constructors: \texttt{Quote}, \texttt{Hole}, and
\texttt{Eval}.

\texttt{Quote} constructs a quotation of type \texttt{epsilon} with
components $t$ and $\alpha$ from a term $t$ of type $\alpha$ that is
is eval-free.  \texttt{Eval} constructs an evaluation of type $\alpha$
with components $t$ and $\alpha$ from a term $t$ of type
\texttt{epsilon} and a type $\alpha$.  \texttt{Hole} is used to
construct ``holes'' of type \texttt{epsilon} in a quasiquotation as
described in~\cite{Farmer18}.  A quotation that contains
holes is a quasiquotation, while a quotation without any holes is a
normal quotation.  The construction of terms has been
modified to allow a hole (of type \texttt{epsilon}) to be used where a
term of some other type is expected.

The external representation of a quotation \texttt{Quote(t,ty)} is
\texttt{Q\_ t \_Q}.  Similarly, the external representation of a hole
\texttt{Hole(t,ty)} is \texttt{H\_ t \_H}.  The external
representation of an evaluation \texttt{Eval(t,ty)} is $\texttt{eval
  t to ty}.$

\subsection{Modification of the HOL Light Proof System}\label{subsec:hlqe-proof-system}

The proof system for {\churchqe} is obtained by extending {\qzero}'s
with additional axioms B1--B13 (see~\ref{subsec:cttqe-proof-system}).
Since {\qzero} and {\HL} are both complete (with respect to the
semantics of Henkin-style general models), {\HL} includes the
reasoning capabilities of the proof system for {\qzero} but not the
reasoning capabilities embodied in the B1--B13 axioms, which must be
implemented in {\HLQE} as follows.  First, the logical constants
defined by Axioms B1--B4, B5, and B7 are defined in {\HLQE} as {\HOL}
functions.  Second, the no junk (B6) and no confusion (B4)
requirements for $\epsilon$ are automatic consequences of defining
\texttt{epsilon} as an inductive type.  Third, Axiom B9 is implemented
directly in the {\HL} code for substitution.  Fourth, Axiom B11.1 is
subsumed by the the BETA rule in {\HL}.  Fifth, the remaining axioms
--- B8, B10, B11.2, B12, and B13 --- are implemented by new rules of
inference in as shown in Table~\ref{tab:axioms}.

%\vspace{-2mm}
\begin{table}
\bc
\begin{tabular}{|ll|}
\hline
${\bf CTT_{qe}}$ \textbf{Axioms}                & \textbf{NewRules of Inference}\\
B8 (Properties of Quotation)                     & \texttt{LAW\_OF\_QUO}\\
B10 (Properties of Evaluation)                   & \texttt{}\\
B10.1                                            & \texttt{VAR\_DISQUO}\\
B10.2                                            & \texttt{CONST\_DISQUO}\\
B10.3                                            & \texttt{APP\_DISQUO}\\
B10.4                                            & \texttt{ABS\_DISQUO}\\
B10.5                                            & \texttt{QUO\_DISQUO}\\
B11.2 (Beta-Reduction for Evaluations)           & \texttt{BETA\_REDUCE\_EVAL}\\
B12 (``Not Free In'' means ``Not Effective In'') & \texttt{NOT\_FREE\_NOT\_EFFECTIVE\_IN}\\
B13 (Beta-Reduction for Function Abstractions)   & \texttt{BETA\_REDUCE\_ABS}\\
\hline
\end{tabular}
\ec
\caption{New Inference Rules in {\HLQE}}\label{tab:axioms} 
\end{table}

%\vspace*{-7mm}
The \texttt{INST} rule of inference is also modified.  This rule
simultaneously substitutes a list of terms for a list of variables in
a sequent.  The substitution function \texttt{vsubst} defined in the
{\HL} kernel is modified so that it works like substitution (via
beta-reduction  rules) does in {\churchqe}.  The main changes are:

\be

  \item A substitution of a term \texttt{t} for a variable \texttt{x}
    in a function abstraction \texttt{Abs(y,s)} is performed as usual
    if (1) \texttt{t} is eval-free and \texttt{x} is not free in
    \texttt{t}, (2) there is a theorem that says \texttt{x} is not
    effective in \texttt{t}, (3) \texttt{s} is eval-free and
    \texttt{x} is not free in \texttt{s}, or (4) there is a theorem
    that says \texttt{x} is not effective in \texttt{s}.  Otherwise,
    if \texttt{s} or \texttt{t} is not eval-free, (5) the function
    abstraction application \texttt{Comb(Abs(x,Abs(y,s)),t)} is
    returned and if \texttt{s} and \texttt{t} are eval-free, (6) the
    variable \texttt{x} is renamed and the substitution is continued.
    When (5) happens, this part of the substitution is finished and
    the user can possibly continue it by applying
    \texttt{BETA\_REDUCE\_ABS}, the rule of inference corresponding to
    Axiom B13.

  \item A substitution of a term \texttt{t} for a variable \texttt{x}
    in a quotation \texttt{Quote(e,ty)} where \texttt{e} does not
    contain any holes (i.e., terms of the form \texttt{Hole(e',ty')})
    returns \texttt{Quote(e,ty)} unchanged (as stated in Axiom B9).
    If \texttt{e} does contain holes, then \texttt{t} is substituted
    for the variable \texttt{x} in the holes in \texttt{Quote(e,ty)}.

  \item A substitution of a term \texttt{t} for a variable \texttt{x}
    in an evaluation \texttt{Eval(e,ty)} returns (1)
    \texttt{Eval(e,ty)} when \texttt{t} is \texttt{x} and (2) the
    function abstraction application
    \texttt{Comb(Abs(x,Eval(e,ty)),t)} otherwise.  (1) is valid by
    Axiom B11.1.  When (2) happens, this part of the substitution is
    finished and the user can possibly continue it by applying
    \texttt{BETA\_REDUCE\_EVAL}, the rule of inference corresponding
    to Axiom B11.2.

\ee

\subsection{Creation of Support Machinery}\label{subsec:machinery}

The {\HLQE} system contains a number of {\HOL} functions, tactics, and
theorems that are useful for reasoning about constructions,
quotations, and evaluations.  An important example is the {\HOL}
function \texttt{isExprType} that implements the {\churchqe} family of
logical constants $\mname{is-expr}_{\epsilon \tarrow o}^{\alpha}$
where $\alpha$ ranges over members of $\sT$.  This function takes
terms $s_1$ and $s_1$ of type \texttt{epsilon} and \texttt{type},
respectively, and returns true iff $s_1$ represents the syntax tree of
a term $t$, $s_2$ represents the syntax tree of a type $\alpha$, and
$t$ is of type $\alpha$.

\subsection{Metatheorems}

We state three important metatheorems about {\HLQE}.  The proofs of
these metatheorems are straightforward but also tedious.  We label the
metatheorems as conjectures since their proofs have not yet been fully
written down.

\begin{conj}\bsp
Every formula provable in {\HL}'s proof system is also provable in
{\HLQE}'s proof system.\esp
\end{conj}

\noindent
\emph{Proof sketch.}  {\HLQE}'s proof system extends {\HL}'s proof
system with new machinery for reasoning about quotations and
evaluations. Thus every {\HL} proof remains valid in {\HLQE}. \hfill
$\Box$

\bigskip

\noindent
Note: All the proofs loaded with the {\HL} system continue to be valid
when loaded in {\HLQE}.  A further test for the future would be to
load a variety of large {\HL} proofs in {\HLQE} to check that their
validity is preserved.

\begin{conj}
The proof system for {\HLQE} is sound for all formulas and complete
for all eval-free formulas.
\end{conj}

\noindent
\emph{Proof sketch.}  The analog of this statement for {\churchqe} is
proved in~\cite{Farmer18}.  It should be possible to prove this
conjecture by just imitating the proof for {\churchqe}.  \hfill $\Box$

\begin{conj}\bsp
{\HLQE} is a model-theoretic conservative extension of {\HL}.\esp
\end{conj}

\noindent
\emph{Proof sketch.}  A model of {\HLQE} is a model of {\HL} with
definitions of the type $\epsilon$ and several constants and
interpretations for the (quasi)quotation and evaluation operators.
These additions do not impinge upon the semantics of {\HL}; hence
every model of {\HL} can be expanded to a model of the {\HLQE}, which
is the meaning of the conjecture.  \hfill $\Box$

\section{Binary Arithmetic} \label{sec:binary-arithmetic}

\section{Related Work} \label{sec:related-work}

\section{Conclusion} \label{sec:conclusion}

\section*{Acknowledgments} 

\bibliography{imps}
\bibliographystyle{plain}

\end{document}
