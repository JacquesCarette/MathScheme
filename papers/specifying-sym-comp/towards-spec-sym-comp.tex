\documentclass[fleqn]{llncs}
\usepackage{latexsym}
\usepackage{amssymb,amsmath}
\usepackage{stmaryrd}
\usepackage{graphicx}
%\usepackage{color}
\usepackage{hyperref}
\usepackage{phonetic}
\usepackage{xargs}
\usepackage[pdftex,dvipsnames]{xcolor}
\usepackage{listings}

\lstset{language=haskell,basicstyle=\ttfamily\small,breaklines=true,showspaces=false,
  showstringspaces=false,breakatwhitespace=true,texcl=true}

\input{towards-spec-sym-comp-def}

\usepackage[colorinlistoftodos,textsize=tiny]{todonotes}
\newcommandx{\unsure}[2][1=]{\todo[linecolor=red,backgroundcolor=red!25,bordercolor=red,#1]{#2}}
\newcommandx{\change}[2][1=]{\todo[linecolor=blue,backgroundcolor=blue!25,bordercolor=blue,#1]{#2}}
\newcommandx{\info}[2][1=]{\todo[linecolor=OliveGreen,backgroundcolor=OliveGreen!25,bordercolor=OliveGreen,#1]{#2}}
\newcommandx{\improvement}[2][1=]{\todo[linecolor=Plum,backgroundcolor=Plum!25,bordercolor=Plum,#1]{#2}}

\title{Towards Specifying Symbolic Computation\thanks{This research is
    supported by NSERC.}}

\author{Jacques Carette and William M. Farmer}

\institute{%
Computing and Software, McMaster University, Canada\\
\url{http://www.cas.mcmaster.ca/~carette}\\
\url{http://imps.mcmaster.ca/wmfarmer}\\[1.5ex]
%21 Feburary 2019
}

\pagestyle{headings}

\iffalse

Examples:

  1. Factoring integers
  2. Change of representation for polynomials
  3. Normalization of rational expressions
  4. Integration by parts
  5. Symbolic differentiation
      o ln/exp example

Example format:

  1. Mathematical task
  2. How the task is performed in Maple
  3. Problems with performing the task
  4. Proposed solution
  5. Specification of the task in CTT_uqe

\fi

\begin{document}

\maketitle

\begin{abstract}
??
\end{abstract}

\iffalse 

\textbf{Keywords:} ??

\fi

\section{Introduction}

\newcommand{\QQ}{\ensuremath{\mathbb{Q}}}
\newcommand{\NRE}{\ensuremath{\mname{normRatExpr}}}
\newcommand{\funQ}[1]{\ensuremath{\LambdaApp x : \QQ \mdot#1}}
\newcommand{\Lang}{\ensuremath{\mathcal{L}}}

\section{Rational Expressions, Rational Functions}

\subsection{The Problem}

Let $e$ be an expression in the language $\Lang$ of the field
$\QQ(x)$, that is, a well-formed expression built from the symbols $x,
0, 1, +, *, -, \phantom{}^{-1}$, elements of $\QQ$ and parentheses (as
necessary).  For greater readability, we will take the liberty of
using fractional notation for $\phantom{}^{-1}$ and the exponential
notation $x^n$ for $x * \cdots * x$ ($n$ times).  $e$ can be something
simple like $\frac{x^4-1}{x^2-1}$ or something more complicated like
\begin{equation*}
\frac{\frac{1-x}{3/2 x^{18} + x + 17}}
     {\frac{1}{9834*x^{19393874}-1/5}}+3*x -\frac{12}{x}.
\end{equation*}
The meaning of $e$, written $\sembrack{e}_{\mathbb{Q}(x)}$, is the
element of $\QQ(x)$ that $e$ denotes.  If $e$ is an expression like
$x/0$, then the meaning of $e$ is undefined and $e$ is said to be
\emph{undefined}.  We will call members of $\Lang$ \emph{rational
  expressions}.  Two rational expressions $e$ and $e'$ are
\emph{equal}, written $e = e'$, if they are both defined and
$\sembrack{e}_{\mathbb{Q}(x)} = \sembrack{e'}_{\mathbb{Q}(x)}$ and are
\emph{quasi-equal}, written $e \simeq e'$, if either $e = e'$ or they
are both undefined.

We are taught that, like for members of $\QQ$ (such as $5/15$), there
is a \emph{normal form} for rational expressions. This is typically
defined to be an expression $p/q$ for two polynomials $p,q \in
\QQ\left[x\right] \subseteq \QQ(x)$ such that $p$ and $q$ are
themselves in polynomial normal form and $\mname{gcd}\left(p,q\right)
= 1$.  The motivation for the latter property is that we usually want
to write $\frac{x^4-1}{x^2-1}$ as $x^2 + 1$ just as we usually want to
write $5/15$ as $1/3$.  Thus, the normal forms of
$\frac{x^4-1}{x^2-1}$ and $\frac{x}{x}$ are $x^2 + 1$ and $1$,
respectively.  This definition of normal form is based on the
characteristic that the elements of the \emph{field of fractions} of a
ring $R$ can be written as quotients $r/s$ of elements of $R$ where
$r_0/s_0 = r_1/s_1$ if and only if $r_0 * s_1 = r_1 * s_0$ in $R$.

Every computer algebra system implements a function that \emph{normalizes}
expressions that denote elements of $\QQ$, $\QQ[x]$, and $\QQ(x)$.  Let {\NRE}
be the name of an algorithm that implements this function for $\QQ(x)$.  It
should certainly have the signature $\NRE : \Lang \rightarrow \Lang$ and
satisfy the invariant $\NRE(e) \simeq e$ --- in other words, be \emph{meaning
preserving} --- for all $e \in \Lang$.  {\NRE} is an example of an SBMA.
\info{SBMA should be defined in the Introduction.}
Note that merely giving the signature of {\NRE} and saying that it is meaning
preserving is a (tremendously) incomplete specification of the
\emph{computational behavior} of {\NRE}.

\iffalse
SBMAs can be difficult to specify since they involve an interplay of
syntax and semantics.  We have to be careful of \emph{which} semantics
is used to guide the syntactic manipulations, as different semantics
for the \textbf{same} expression can be inconsistent.
\fi

A rational expression $e \in \QQ(x)$ can be interpreted as a \emph{function}
$f = \LambdaApp x : \QQ \mdot e$.  Such functions are typically called
\emph{rational functions}.  However, equality in $\QQ(x)$ and in
$\QQ\rightarrow\QQ$ differ.  For example, one might think that the rational
functions $\funQ{x/x}$ and $\funQ{1}$ should be equal since $x/x$ and
$1$ are equal as rational expressions, or that $\funQ{1/x - 1/x}$ and
$\funQ{0}$ are too. But they are not since both
$\funQ{x/x}$ and $\funQ{1/x - 1/x}$ are undefined at 0, while both
$\funQ{1}$ and $\funQ{0}$ are defined everywhere.  Calling $\funQ{x/x}$
and $\funQ{1}$ is frequently justified by invoking the concept of
\emph{removable singulaties} -- but this reasoning is less clear as a
justification that $\funQ{1/x - 1/x}$ and $\funQ{0}$ are equal.

Why is this an issue? Mainly because CAS make little distinction between
the two.  For example, one can always \emph{evaluate} an expression
for its free variables, or even convert it to a function. In
Maple\footnote{Mathematica has similar commands}, these are done
respectively via \texttt{eval(e, x = 0)} and
\textit{unapply(e, x)}.  We can exhibit the problematic behaviour as
follows:
\todo{insert some Maple code with output here}
In fact, there is an even more pervasive, one could even say
\emph{obnoxious} way of doing this: as the underlying language is
\emph{imperative}, it is possible to do
\begin{verbatim}
e := (x^4-1)/(x^2-1);
# many, many more lines of 'code'
x := 1;
# try to use 'e'
\end{verbatim}

Hence, if an expression $e$ is interpreted as a
function, then it is not valid to simplify the function by applying
{\NRE} to $e$, but CASs let the user do exactly this because usually
there is no distinction made between $e$ as a rational expression and
$e$ as representing a rational function, as shown above.  We need to know the
\emph{mathematical meaning} of {\NRE} applied to rational functions to
be able to avoid unsound applications of {\NRE}.

\unsure{I don't know why we need to say this: Of course, given some
  symbol $y$, $f(y)$ \textbf{is} in $\Lang$.}

We are thus interested in the following questions:

\be

  \item What should the specification of the computational behavior of
    {\NRE} be?

  \item What is the mathematical meaning of {\NRE} be when {\NRE} is
    applied to the body of a rational function?

  \item What features of a logic are needed to express
    {\NRE}'s specification and mathematical meaning?

  \item What features of a logic would make expressing {\NRE}'s
    specification and mathematical meaning relatively straightforward?

\ee

\subsection{The Specification of {\NRE}}

\subsection{The Mathematical Meaning of {\NRE}}

\section{Conclusion}

\bibliography{imps}
\bibliographystyle{splncs04}

\setcounter{tocdepth}{1}
\listoftodos
\setcounter{tocdepth}{0}

\end{document}
