\documentclass[fleqn]{llncs}
\usepackage{latexsym}
\usepackage{amssymb,amsmath}
\usepackage{stmaryrd}
\usepackage{graphicx}
%\usepackage{color}
\usepackage{hyperref}
\usepackage{phonetic}
\usepackage{xargs}
\usepackage[pdftex,dvipsnames]{xcolor}
\usepackage{listings}

\lstset{language=haskell,basicstyle=\ttfamily\small,breaklines=true,showspaces=false,
  showstringspaces=false,breakatwhitespace=true,texcl=true}

\newcommand{\additionu}[1]{\textcolor{blue}{#1}}
\newcommand{\additionuqe}[1]{\textcolor{red}{#1}}

\renewcommand{\labelenumii}{\theenumii.}
\newcommand{\be}{\begin{enumerate}}
\newcommand{\ee}{\end{enumerate}}
\newcommand{\bi}{\begin{itemize}}
\newcommand{\ei}{\end{itemize}}
\newcommand{\bc}{\begin{center}}
\newcommand{\ec}{\end{center}}
\newcommand{\bsp}{\begin{sloppypar}}
\newcommand{\esp}{\end{sloppypar}}

\newtheorem{thm}{Theorem}[subsection]
\newtheorem{cor}[thm]{Corollary}
\newtheorem{lem}[thm]{Lemma}
\newtheorem{prop}[thm]{Proposition}
\newtheorem{rem}[thm]{Remark}
\newtheorem{eg}[thm]{Example}
\newtheorem{df}[thm]{Definition}

\newtheorem{thm1}{Theorem}[section]
\newtheorem{cor1}[thm1]{Corollary}
\newtheorem{lem1}[thm1]{Lemma}
\newtheorem{prop1}[thm1]{Proposition}
\newtheorem{rem1}[thm1]{Remark}
\newtheorem{eg1}[thm1]{Example}
\newtheorem{df1}[thm1]{Definition}

%\newtheorem{note}{Note}

%\newenvironment{proof}{\par\noindent{\bf Proof\sglsp}}{\hfill$\Box$}

\newcommand{\sglsp}{\ }
\newcommand{\dblsp}{\ \ }

\newcommand{\sA}{\mbox{$\cal A$}}
\newcommand{\sB}{\mbox{$\cal B$}}
\newcommand{\sC}{\mbox{$\cal C$}}
\newcommand{\sD}{\mbox{$\cal D$}}
\newcommand{\sE}{\mbox{$\cal E$}}
\newcommand{\sF}{\mbox{$\cal F$}}
\newcommand{\sG}{\mbox{$\cal G$}}
\newcommand{\sH}{\mbox{$\cal H$}}
\newcommand{\sI}{\mbox{$\cal I$}}
\newcommand{\sJ}{\mbox{$\cal J$}}
\newcommand{\sK}{\mbox{$\cal K$}}
\newcommand{\sL}{\mbox{$\cal L$}}
\newcommand{\sM}{\mbox{$\cal M$}}
\newcommand{\sN}{\mbox{$\cal N$}}
\newcommand{\sO}{\mbox{$\cal O$}}
\newcommand{\sP}{\mbox{$\cal P$}}
\newcommand{\sQ}{\mbox{$\cal Q$}}
\newcommand{\sR}{\mbox{$\cal R$}}
\newcommand{\sS}{\mbox{$\cal S$}}
\newcommand{\sT}{\mbox{$\cal T$}}
\newcommand{\sU}{\mbox{$\cal U$}}
\newcommand{\sV}{\mbox{$\cal V$}}
\newcommand{\sW}{\mbox{$\cal W$}}
\newcommand{\sX}{\mbox{$\cal X$}}
\newcommand{\sY}{\mbox{$\cal Y$}}
\newcommand{\sZ}{\mbox{$\cal Z$}}

\renewcommand{\phi}{\varphi}

\newcommand{\churchqe}{$\mbox{\sc ctt}_{\rm qe}$}
\newcommand{\churchuqe}{$\mbox{\sc ctt}_{\rm uqe}$}
\newcommand{\churcheps}{$\mbox{\sc ctt}_\epsilon$}
\newcommand{\qzero}{${\cal Q}_0$}
\newcommand{\qzerou}{${\cal Q}^{\rm u}_{0}$}
\newcommand{\qzerouqe}{${\cal Q}^{\rm uqe}_{0}$}
\newcommand{\iotaAlt}{\mbox{\it \i}}
\newcommand{\iotaAltS}{\mbox{{\scriptsize \it \i}}}
\newcommand{\pfsys}{${\cal P}$}
\newcommand{\pfsysu}{${\cal P}^{\rm u}$}
\newcommand{\pfsysuq}{${\cal P}^{\rm uq}$}
\newcommand{\pfsysuqeplus}{$\overline{{\cal P}^{\rm uqe}}$}
\newcommand{\pfsysuqe}{${\cal P}^{\rm uqe}$}
\newcommand{\sub}{\d{\sf S}}
\newcommand{\HOL}{$\mbox{\rm HOL}$}
\newcommand{\HL}{$\mbox{\rm HOL Light}$}
\newcommand{\HLQE}{$\mbox{\rm HOL Light QE}$}
\newcommand{\OCAML}{$\mbox{\rm OCaml}$}
\newcommand{\MMT}{$\mbox{\sc Mmt}$}

\newcommand{\seq}[1]{{\langle #1 \rangle}}
\newcommand{\set}[1]{{\{ #1 \}}}
\newcommand{\sembrack}[1]{\llbracket#1\rrbracket}
\newcommand{\synbrack}[1]{\ulcorner#1\urcorner}
\newcommand{\commabrack}[1]{\lfloor#1\rfloor}
\newcommand{\mname}[1]{\mbox{\sf #1}}
\newcommand{\mcolon}{\mathrel:}
\newcommand{\mdot}{\mathrel.}
\newcommand{\tarrow}{\rightarrow}
\newcommand{\LambdaApp}{\lambda\,}
\newcommand{\Neg}{\neg}
\newcommand{\NegAlt}{{\sim}}
\ifdefined \And 
\renewcommand{\And}{\wedge}
\else
\newcommand{\And}{\wedge}
\fi
\newcommand{\Implies}{\supset}
\newcommand{\Or}{\vee}
\newcommand{\Iff}{\equiv}
\newcommand{\Forall}{\forall}
\newcommand{\ForallApp}{\forall\,}
\newcommand{\Forsome}{\exists}
\newcommand{\ForsomeApp}{\exists\,}
\newcommand{\ForsomeUniqueApp}{\exists\,!\,}
\newcommand{\Iota}{\mbox{\rm I}}
\newcommand{\IotaApp}{\mbox{\rm I}\,}
\newcommand{\IsDef}{\downarrow}
\newcommand{\IsUndef}{\uparrow}
\newcommand{\Equal}{=}
\newcommand{\QuasiEqual}{\simeq}
\newcommand{\Undefined}{\bot}
\newcommand{\If}{\mname{if}}
\newcommand{\IsDefApp}{\!\IsDef}
\newcommand{\IsUndefApp}{\!\IsUndef}
\newcommand{\invertediota}{\rotatebox[origin=c]{180}{$\iota$}}
\newcommand{\pf}{\mbox{\sc pf}}
\newcommand{\pfstar}{$\pf^{\ast}$}
\newcommand{\imps}{\mbox{\sc imps}}
\newcommand{\tps}{\mbox{\sc tps}}
\newcommand{\hol}{\mbox{\sc hol}}
\newcommand{\pvs}{\mbox{\sc pvs}}
\newcommand{\lutins}{\mbox{\sc lutins}}
\newcommand{\sttwu}{\mbox{{\sc stt}{\small w}{\sc u}}}
\newcommand{\modelsa}{\mathrel{\models^{\rm ef}\!}}
\newcommand{\modelsb}{\mathrel{\overline{\models^{\rm ef}}\!}}
\newcommand{\modelsn}{\mathrel{\models_{\rm n}}}
\newcommand{\modelsna}{\mathrel{\models^{\rm ef\!}_{\rm n}}}
\newcommand{\modelsnb}{\mathrel{\overline{\models^{\ast}_{\rm n}}}}
\newcommand{\proves}[2]{#1 \vdash #2}
\newcommand{\provesa}[2]{#1 \mathrel{\vdash^{\rm ef}\!} #2}
\newcommand{\provesb}[2]{#1 \mathrel{\overline{\vdash^{\rm ef}}\!} #2}
\newcommand{\wff}[1]{{\rm wff}_{#1}}
\newcommand{\xwff}[1]{{\rm xwff}_{#1}}
\newcommand{\wffs}[1]{{\rm wffs}_{#1}}
\newcommand{\xwffs}[1]{{\rm xwffs}_{#1}}

\newcommand{\TRUE}{\mbox{{\sc t}}}
\newcommand{\FALSE}{\mbox{{\sc f}}}

\newcommand{\nbg}{\mbox{\sc nbg}}


\usepackage[colorinlistoftodos,textsize=tiny]{todonotes}
\newcommandx{\unsure}[2][1=]{\todo[linecolor=red,backgroundcolor=red!25,bordercolor=red,#1]{#2}}
\newcommandx{\change}[2][1=]{\todo[linecolor=blue,backgroundcolor=blue!25,bordercolor=blue,#1]{#2}}
\newcommandx{\info}[2][1=]{\todo[linecolor=OliveGreen,backgroundcolor=OliveGreen!25,bordercolor=OliveGreen,#1]{#2}}
\newcommandx{\improvement}[2][1=]{\todo[linecolor=Plum,backgroundcolor=Plum!25,bordercolor=Plum,#1]{#2}}

\newcommand{\QQ}{\ensuremath{\mathbb{Q}}}
\newcommand{\NRE}{\ensuremath{\mname{normRatExpr}}}
\newcommand{\NRF}{\ensuremath{\mname{normRatFun}}}
\newcommand{\funQ}[1]{\ensuremath{\LambdaApp x : \QQ \mdot #1}}
\newcommand{\Lang}{\ensuremath{\mathcal{L}}}
\newcommand{\Langp}{\ensuremath{\mathcal{L}'}}

\title{Towards Specifying Symbolic Computation\thanks{This research is
    supported by NSERC.}}

\author{Jacques Carette and William M. Farmer}

\institute{%
Computing and Software, McMaster University, Canada\\
\url{http://www.cas.mcmaster.ca/~carette}\\
\url{http://imps.mcmaster.ca/wmfarmer}\\[1.5ex]
%21 Feburary 2019
}

\pagestyle{headings}

\iffalse

Examples:

  1. Factoring integers
  2. Change of representation for polynomials
  3. Normalization of rational expressions
  4. Integration by parts
  5. Symbolic differentiation
      o ln/exp example

Example format:

  1. Mathematical task
  2. How the task is performed in Maple
  3. Problems with performing the task
  4. Proposed solution
  5. Specification of the task in CTT_uqe

\fi

\begin{document}

\maketitle

\begin{abstract}
??
\end{abstract}

\iffalse 

\textbf{Keywords:} ??

\fi

\section{Introduction}

\section{Background}

Let $e$ be a mathematical expression and $D$ be a domain of
mathematical values.  We say \emph{$e$ is defined in $D$} if $e$
denotes an element in $D$.  When $e$ is defined in $D$, the
\emph{value of $e$ in $D$}, written $\mname{val}_D(e)$, is the element
in $D$ that $e$ denotes.  When $e$ is undefined in $D$, the value of
$e$ in $D$ and $\mname{val}_D(e)$ are undefined.  Two expressions $e$
and $e'$ are \emph{equal in $D$}, written $e =_D e'$, if $e$ and $e'$
are defined in $D$ and $\mname{val}_D(e) = \mname{val}_D(e')$ and are
\emph{quasi-equal in $D$}, written $e \simeq_D e'$, if either $e =_D
e'$ or $e$ and $e'$ are both undefined in $D$.

\iffalse
SBMAs can be difficult to specify since they involve an interplay of
syntax and semantics.  We have to be careful of \emph{which} semantics
is used to guide the syntactic manipulations, as different semantics
for the \textbf{same} expression can be inconsistent.

We are thus interested in the following questions:

\be

  \item What should the specification of the computational behavior of
    {\NRE} be?

  \item What is the mathematical meaning of {\NRE} be when {\NRE} is
    applied to the body of a rational function?

  \item What features of a logic are needed to express
    {\NRE}'s specification and mathematical meaning?

  \item What features of a logic would make expressing {\NRE}'s
    specification and mathematical meaning relatively straightforward?

\ee
\fi




\section{Rational Expressions, Rational Functions}

\subsection{Rational Expressions}

Let $e$ be an expression in the language $\Lang$ of the field
$\QQ(x)$, that is, a well-formed expression built from the symbols $x,
0, 1, +, *, -, \phantom{}^{-1}$, elements of $\QQ$ and parentheses (as
necessary).  For greater readability, we will take the liberty of
using fractional notation for $\phantom{}^{-1}$ and the exponential
notation $x^n$ for $x * \cdots * x$ ($n$ times).  $e$ can be something
simple like $\frac{x^4-1}{x^2-1}$ or something more complicated like
\begin{equation*}
\frac{\frac{1-x}{3/2 x^{18} + x + 17}}
     {\frac{1}{9834*x^{19393874}-1/5}}+3*x -\frac{12}{x}.
\end{equation*}
We assume that $\QQ \subseteq \QQ[x] \subseteq \QQ(x)$ so that the
field of rational numbers and the ring of polynomials in $x$ are
included in $\QQ(x)$.  The expressions in $\Lang$ are intended to
denote elements in $\QQ(x)$.  Of course, expressions like $x/0$ are
undefined in $\QQ(x)$.  We will call members of $\Lang$ \emph{rational
  expressions (over $\QQ$)}.

We are taught that, like for members of $\QQ$ (such as $5/15$), there
is a \emph{normal form} for rational expressions. This is typically
defined to be a rational expression $p/q$ for two polynomials $p,q \in \QQ[x]$
such that $p$ and $q$ are themselves in polynomial normal form and
$\mname{gcd}(p,q) = 1$.  The motivation for the latter
property is that we usually want to write $\frac{x^4-1}{x^2-1}$ as
$x^2 + 1$ just as we usually want to write $5/15$ as $1/3$.  Thus, the
normal forms of $\frac{x^4-1}{x^2-1}$ and $\frac{x}{x}$ are $x^2 + 1$
and $1$, respectively.  This definition of normal form is based on the
characteristic that the elements of the \emph{field of fractions} of a
ring $R$ can be written as quotients $r/s$ of elements of $R$ where
$r_0/s_0 = r_1/s_1$ if and only if $r_0 * s_1 = r_1 * s_0$ in $R$.

Every computer algebra system implements a function that
\emph{normalizes} expressions that denote elements of $\QQ(x)$
(including elements of $\QQ$ and $\QQ[x]$).  \change{Unfortunately
that statement is not quite right, because normalization in a CAS merely means
that the result can checked to be 0 (or not) in O(1) time. This leads
to different normalizations for all 3, implemented in 3 different
functions. It turns out that, in the univariate case, they correspond,
but already for 2 variables things are different.}
\info{I think you might be conflating what CAS people call normal and
canonical. Normal just means O(1) zero-testing, while canonical means
a = b iff C(a) = C(b) with the later = being O(1) because of 
hash-consing}
Let {\NRE} be the name of
the algorithm that implements this normalization function on $\Lang$.
Thus the signature of $\NRE$ is $\Lang \rightarrow \Lang$ and the
specification of $\NRE$ is that, for all $e \in \Lang$, (A) $\NRE(e)$
is a normal form and (B) $e \simeq_{\QQ(x)} \NRE(e)$.  $\NRE$ is an
example of an SBMA.  (A) is the syntactic component of its
specification, and (B) is the semantic component.
\change{in the above, you never actually define what a normal form is!}

\subsection{Rational Functions}

Let $\Langp$ be the set of expressions of the form $\LambdaApp x : \QQ
\mdot e$ where $e \in \Lang$.  We will call members of $\Langp$
\emph{rational functions (over $\QQ$)}.  That is, a rational function
is a lambda expression whose body is a rational expression.  

If $f_i = \funQ{e_i}$ are rational functions for $i=1,2$, one might
think that $f_1 =_{\QQ \rightarrow \QQ} f_2$ if $e_1 =_{\QQ(x)} e_2$.
But this is not the case.  For example, the rational functions
$\funQ{x/x}$ and $\funQ{1}$ are not equal since $\funQ{x/x}$ is
undefined at 0 while $\funQ{1}$ is defined everywhere.  But $x/x
=_{\QQ(x)} 1$! Similarly, $\funQ{(1/x - 1/x)} \not=_{\QQ \rightarrow
  \QQ} \funQ{0}$ and $(1/x - 1/x) =_{\QQ(x)} 0$.  Note that, in some
contexts, we might want to say that $\funQ{x/x}$ and $\funQ{1}$ do
indeed denote the same function by invoking the concept of
\emph{removable singularities}.
%
\unsure{I don't see why this reasoning is less clear as a
  justification that $\funQ{(1/x - 1/x)}$ and $\funQ{0}$ are equal.}

As we have just seen, we cannot normalize a rational function by
normalizing its body, but we can normalize rational functions if we
are careful not to remove points of undefinedness.  Let a
\emph{quasinormal form} be a rational expression $p/q$ for two
polynomials $p,q \in \QQ[x]$ such that $p$ and $q$ are themselves in
polynomial normal form and there is no irreducible polynomial $r \in
\QQ[x]$ of degree $\ge 2$ that divides both $p$ and $q$.  We can then
normalize a rational function by quasinormalizing its body.  Let
{\NRF} be the name of the algorithm that implements this normalization
function on $\Langp$.  Thus the signature of $\NRF$ is $\Langp
\rightarrow \Langp$ and the specification of $\NRF$ is that, for all
$\funQ{e} \in \Langp$, (A) $\NRF(\funQ{e}) = \funQ{e'}$ where $e'$ is
a quasinormal form and (B) $\funQ{e} \simeq_{\QQ \rightarrow \QQ}
\NRF(\funQ{e})$.  $\NRF$ is another example of an SBMA.  (A) is the
syntactic component of its specification, and (B) is the semantic
component.
\unsure{Why those conditions on $r$? It is ok, over $\QQ(x)$, to remove
a common factor of $x^2+1$. Or even $x^2-2$ !}

\subsection{The Problem Here}

So why are we concerned about rational expressions and rational
functions?  The reason is that computer algebra systems make little
distinction between the two: a rational expression can be interpreted
sometimes as a rational expression and sometimes as a rational
function.  For example, one can always \emph{evaluate} an expression
by assigning values to its free variables or even convert it to a
function.  In Maple\footnote{Mathematica has similar commands.}, these
are done respectively via \texttt{eval(e, x = 0)} and
\texttt{unapply(e, x)}.  We can exhibit the problematic behaviour as
follows: \todo{insert some Maple code with output here} In fact, there
is an even more pervasive, one could even say \emph{obnoxious}, way of
doing this: as the underlying language is \emph{imperative}, it is
possible to do:
\begin{verbatim}
  e := (x^4-1)/(x^2-1);
  # many, many more lines of 'code'
  x := 1;
   try to use 'e'
\end{verbatim}

Hence, if an expression $e$ is interpreted as a function, then it is
not valid to simplify the function by applying {\NRE} to $e$, but
computer algebra systems let the user do exactly this because usually
there is no distinction made between $e$ as a rational expression and
$e$ as representing a rational function, as we have already mentioned.

To avoid unsound applications of {\NRE}, {\NRF}, and other SBMAs in
mathematical systems, we need to carefully, if not formally, specify
what these algorithms are intended to do.  This is not a
straightforward task to do in a traditional logic since SBMAs involve
an interplay of syntax and semantics and algorithms like {\NRE} and
{\NRF} are very sensitive to definedness considerations.  In the next
subsection we will show how these two algorithms can be specified in a
version of formal logic with undefinedness, quotation, and evaluation.

\unsure{I don't know why we need to say this: ``Of course, given some
  symbol $y$, $f(y)$ \textbf{is} in $\Lang$.''}

\subsection{The Formal Specification of {\NRE} and {\NRF}}

\section{Related Work}

\section{Conclusion}

\bibliography{imps}
\bibliographystyle{splncs04}

\setcounter{tocdepth}{1}
\listoftodos
\setcounter{tocdepth}{0}

\end{document}
