\documentclass[fleqn]{llncs}
\usepackage{latexsym}
\usepackage{amssymb,amsmath}
\usepackage{stmaryrd}
\usepackage{graphicx}
%\usepackage{color}
\usepackage{hyperref}
\usepackage{phonetic}
\usepackage{xargs}
\usepackage[pdftex,dvipsnames]{xcolor}
\usepackage{listings}

\lstset{language=haskell,basicstyle=\ttfamily\small,breaklines=true,showspaces=false,
  showstringspaces=false,breakatwhitespace=true,texcl=true}

\newcommand{\additionu}[1]{\textcolor{blue}{#1}}
\newcommand{\additionuqe}[1]{\textcolor{red}{#1}}

\renewcommand{\labelenumii}{\theenumii.}
\newcommand{\be}{\begin{enumerate}}
\newcommand{\ee}{\end{enumerate}}
\newcommand{\bi}{\begin{itemize}}
\newcommand{\ei}{\end{itemize}}
\newcommand{\bc}{\begin{center}}
\newcommand{\ec}{\end{center}}
\newcommand{\bsp}{\begin{sloppypar}}
\newcommand{\esp}{\end{sloppypar}}

\newtheorem{thm}{Theorem}[subsection]
\newtheorem{cor}[thm]{Corollary}
\newtheorem{lem}[thm]{Lemma}
\newtheorem{prop}[thm]{Proposition}
\newtheorem{rem}[thm]{Remark}
\newtheorem{eg}[thm]{Example}
\newtheorem{df}[thm]{Definition}

\newtheorem{thm1}{Theorem}[section]
\newtheorem{cor1}[thm1]{Corollary}
\newtheorem{lem1}[thm1]{Lemma}
\newtheorem{prop1}[thm1]{Proposition}
\newtheorem{rem1}[thm1]{Remark}
\newtheorem{eg1}[thm1]{Example}
\newtheorem{df1}[thm1]{Definition}

%\newtheorem{note}{Note}

%\newenvironment{proof}{\par\noindent{\bf Proof\sglsp}}{\hfill$\Box$}

\newcommand{\sglsp}{\ }
\newcommand{\dblsp}{\ \ }

\newcommand{\sA}{\mbox{$\cal A$}}
\newcommand{\sB}{\mbox{$\cal B$}}
\newcommand{\sC}{\mbox{$\cal C$}}
\newcommand{\sD}{\mbox{$\cal D$}}
\newcommand{\sE}{\mbox{$\cal E$}}
\newcommand{\sF}{\mbox{$\cal F$}}
\newcommand{\sG}{\mbox{$\cal G$}}
\newcommand{\sH}{\mbox{$\cal H$}}
\newcommand{\sI}{\mbox{$\cal I$}}
\newcommand{\sJ}{\mbox{$\cal J$}}
\newcommand{\sK}{\mbox{$\cal K$}}
\newcommand{\sL}{\mbox{$\cal L$}}
\newcommand{\sM}{\mbox{$\cal M$}}
\newcommand{\sN}{\mbox{$\cal N$}}
\newcommand{\sO}{\mbox{$\cal O$}}
\newcommand{\sP}{\mbox{$\cal P$}}
\newcommand{\sQ}{\mbox{$\cal Q$}}
\newcommand{\sR}{\mbox{$\cal R$}}
\newcommand{\sS}{\mbox{$\cal S$}}
\newcommand{\sT}{\mbox{$\cal T$}}
\newcommand{\sU}{\mbox{$\cal U$}}
\newcommand{\sV}{\mbox{$\cal V$}}
\newcommand{\sW}{\mbox{$\cal W$}}
\newcommand{\sX}{\mbox{$\cal X$}}
\newcommand{\sY}{\mbox{$\cal Y$}}
\newcommand{\sZ}{\mbox{$\cal Z$}}

\renewcommand{\phi}{\varphi}

\newcommand{\churchqe}{$\mbox{\sc ctt}_{\rm qe}$}
\newcommand{\churchuqe}{$\mbox{\sc ctt}_{\rm uqe}$}
\newcommand{\churcheps}{$\mbox{\sc ctt}_\epsilon$}
\newcommand{\qzero}{${\cal Q}_0$}
\newcommand{\qzerou}{${\cal Q}^{\rm u}_{0}$}
\newcommand{\qzerouqe}{${\cal Q}^{\rm uqe}_{0}$}
\newcommand{\iotaAlt}{\mbox{\it \i}}
\newcommand{\iotaAltS}{\mbox{{\scriptsize \it \i}}}
\newcommand{\pfsys}{${\cal P}$}
\newcommand{\pfsysu}{${\cal P}^{\rm u}$}
\newcommand{\pfsysuq}{${\cal P}^{\rm uq}$}
\newcommand{\pfsysuqeplus}{$\overline{{\cal P}^{\rm uqe}}$}
\newcommand{\pfsysuqe}{${\cal P}^{\rm uqe}$}
\newcommand{\sub}{\d{\sf S}}
\newcommand{\HOL}{$\mbox{\rm HOL}$}
\newcommand{\HL}{$\mbox{\rm HOL Light}$}
\newcommand{\HLQE}{$\mbox{\rm HOL Light QE}$}
\newcommand{\OCAML}{$\mbox{\rm OCaml}$}
\newcommand{\MMT}{$\mbox{\sc Mmt}$}

\newcommand{\seq}[1]{{\langle #1 \rangle}}
\newcommand{\set}[1]{{\{ #1 \}}}
\newcommand{\sembrack}[1]{\llbracket#1\rrbracket}
\newcommand{\synbrack}[1]{\ulcorner#1\urcorner}
\newcommand{\commabrack}[1]{\lfloor#1\rfloor}
\newcommand{\mname}[1]{\mbox{\sf #1}}
\newcommand{\mcolon}{\mathrel:}
\newcommand{\mdot}{\mathrel.}
\newcommand{\tarrow}{\rightarrow}
\newcommand{\LambdaApp}{\lambda\,}
\newcommand{\Neg}{\neg}
\newcommand{\NegAlt}{{\sim}}
\ifdefined \And 
\renewcommand{\And}{\wedge}
\else
\newcommand{\And}{\wedge}
\fi
\newcommand{\Implies}{\supset}
\newcommand{\Or}{\vee}
\newcommand{\Iff}{\equiv}
\newcommand{\Forall}{\forall}
\newcommand{\ForallApp}{\forall\,}
\newcommand{\Forsome}{\exists}
\newcommand{\ForsomeApp}{\exists\,}
\newcommand{\ForsomeUniqueApp}{\exists\,!\,}
\newcommand{\Iota}{\mbox{\rm I}}
\newcommand{\IotaApp}{\mbox{\rm I}\,}
\newcommand{\IsDef}{\downarrow}
\newcommand{\IsUndef}{\uparrow}
\newcommand{\Equal}{=}
\newcommand{\QuasiEqual}{\simeq}
\newcommand{\Undefined}{\bot}
\newcommand{\If}{\mname{if}}
\newcommand{\IsDefApp}{\!\IsDef}
\newcommand{\IsUndefApp}{\!\IsUndef}
\newcommand{\invertediota}{\rotatebox[origin=c]{180}{$\iota$}}
\newcommand{\pf}{\mbox{\sc pf}}
\newcommand{\pfstar}{$\pf^{\ast}$}
\newcommand{\imps}{\mbox{\sc imps}}
\newcommand{\tps}{\mbox{\sc tps}}
\newcommand{\hol}{\mbox{\sc hol}}
\newcommand{\pvs}{\mbox{\sc pvs}}
\newcommand{\lutins}{\mbox{\sc lutins}}
\newcommand{\sttwu}{\mbox{{\sc stt}{\small w}{\sc u}}}
\newcommand{\modelsa}{\mathrel{\models^{\rm ef}\!}}
\newcommand{\modelsb}{\mathrel{\overline{\models^{\rm ef}}\!}}
\newcommand{\modelsn}{\mathrel{\models_{\rm n}}}
\newcommand{\modelsna}{\mathrel{\models^{\rm ef\!}_{\rm n}}}
\newcommand{\modelsnb}{\mathrel{\overline{\models^{\ast}_{\rm n}}}}
\newcommand{\proves}[2]{#1 \vdash #2}
\newcommand{\provesa}[2]{#1 \mathrel{\vdash^{\rm ef}\!} #2}
\newcommand{\provesb}[2]{#1 \mathrel{\overline{\vdash^{\rm ef}}\!} #2}
\newcommand{\wff}[1]{{\rm wff}_{#1}}
\newcommand{\xwff}[1]{{\rm xwff}_{#1}}
\newcommand{\wffs}[1]{{\rm wffs}_{#1}}
\newcommand{\xwffs}[1]{{\rm xwffs}_{#1}}

\newcommand{\TRUE}{\mbox{{\sc t}}}
\newcommand{\FALSE}{\mbox{{\sc f}}}

\newcommand{\nbg}{\mbox{\sc nbg}}


\usepackage[colorinlistoftodos,textsize=tiny]{todonotes}
\newcommandx{\unsure}[2][1=]{\todo[linecolor=red,backgroundcolor=red!25,bordercolor=red,#1]{#2}}
\newcommandx{\change}[2][1=]{\todo[linecolor=blue,backgroundcolor=blue!25,bordercolor=blue,#1]{#2}}
\newcommandx{\info}[2][1=]{\todo[linecolor=OliveGreen,backgroundcolor=OliveGreen!25,bordercolor=OliveGreen,#1]{#2}}
\newcommandx{\improvement}[2][1=]{\todo[linecolor=Plum,backgroundcolor=Plum!25,bordercolor=Plum,#1]{#2}}

\title{Towards Specifying Symbolic Computation\thanks{This research is
    supported by NSERC.}}

\author{Jacques Carette and William M. Farmer}

\institute{%
Computing and Software, McMaster University, Canada\\
\url{http://www.cas.mcmaster.ca/~carette}\\
\url{http://imps.mcmaster.ca/wmfarmer}\\[1.5ex]
%21 Feburary 2019
}

\pagestyle{headings}

\iffalse

Examples:

  1. Factoring integers
  2. Change of representation for polynomials
  3. Normalization of rational expressions
  4. Integration by parts
  5. Symbolic differentiation
      o ln/exp example

Example format:

  1. Mathematical task
  2. How the task is performed in Maple
  3. Problems with performing the task
  4. Proposed solution
  5. Specification of the task in CTT_uqe

\fi

\begin{document}

\maketitle

\begin{abstract}
??
\end{abstract}

\iffalse 

\textbf{Keywords:} ??

\fi

\section{Introduction}

\section{Simplifying Expressions denoting Rational Functions}

\subsection{The Problem}

Let $f = \LambdaApp x : \mathbb{Q} \mdot R(x)$ be an expression that
denotes a function of type $\mathbb{Q} \tarrow \mathbb{Q}$.
Furthermore, let $R(x)$, as a syntactic expression, denote a member of
the field $\mathbb{Q}(x)$ consisting of \emph{rational expressions} of
the form $\frac{P(x)}{Q(x)}$ where $P(x)$ and $Q(x)$ are polynomials
in $x$.  A function that can be represented by an expression like $f$
whose body is a rational expression is called a \emph{rational
  function}.

$R(x)$ could be a very complicated expression build from the primitive
components of the field $\mathbb{Q}(x)$: $0,1,+,*,-,\phantom{}^{-1}$.
An obvious way to simplify $f$ would be to simplify $R(x)$
syntactically as a rational expression in a meaning-preserving way.
This operation, which we will call \mname{simpRatFun}, is an example
of a \emph{syntax-based mathematical algorithm (SBMA)}.  As an SBMA,
\mname{simpRatFun} works by manipulating syntactic expressions in a
mathematically meaningful way.

SBMAs can be difficult to specify since they involve an interplay of
syntax and semantics.  We are interested in the following three
questions:

\be

  \item What should be the specification of \mname{simpRatFun}?

  \item Can this specification be expressed in traditional logic?

  \item How would this specification be expressed in a logic with
    undefinedness, quotation, and evaluation?

\ee

\subsection{A Naive Specification}

Let $R(x)$ be an expression that denotes a member of $\mathbb{Q}(x)$.
The \emph{normal form} of $R(x)$ is an expression $R'(x)$ of the form
$\frac{P(x)}{Q(x)}$ such that $P(x)$ and $Q(x)$ are polynomials in
standard form, $\frac{P(x)}{Q(x)}$ is in lowest terms, and $R(x)$ and
$R'(x)$ both denote the same member of $\mathbb{Q}(x)$.  For example,
the normal form of \[\frac{x^2 + 2x - 1}{x^2 - 1} - \frac{x}{x - 1}\]
is \[\frac{1}{x + 1}.\]

If we do not think too hard, we might be tempted to specify
\mname{simpRatFun} as follows: 

\bi

  \item[] For all expressions $f = \LambdaApp x : \mathbb{Q} \mdot
    R(x)$ of type $\mathbb{Q} \tarrow \mathbb{Q}$ where the expression
    $R(x)$ denotes a member of $\mathbb{Q}(x)$, $\mname{simpRatFun}(f)
    = \LambdaApp x : \mathbb{Q} \mdot R'(x)$ where $R'(x)$ is the
    normal form of $R(x)$.

\ei
%
Hence \mname{simpRatFun} applied to \[\LambdaApp x : \mathbb{Q} \mdot
\frac{x^2 + 2x - 1}{x^2 - 1} - \frac{x}{x - 1}\] should return 
\[\LambdaApp x : \mathbb{Q} \mdot \frac{1}{x + 1}.\]

This specification is essentially the same as Maple's \mname{normal}
operation that reduces the expression $\frac{x^2 - 2x - 1}{x^2 - 1} -
\frac{x}{x - 1}$ to $\frac{1}{x + 1}$.

Is this specification of \mname{simpRatFun} correct?  If so, $f$ and
$\mname{simpRatFun}(f)$ should denote the same function of type
$\mathbb{Q} \tarrow \mathbb{Q}$ for all expressions $f$ that denote
rational functions.  Unfortunately, this is not the case.  Let \[f =
\LambdaApp x : \mathbb{Q} \mdot \frac{x^2 + 2x - 1}{x^2 - 1} -
\frac{x}{x - 1}.\]  Then $f(1)$ is obviously undefined, but
$\mname{simpRatFun}(f)(1) = 1/2$.  Hence \mname{simpRatFun} is --- as
specified --- not meaning preserving.  What went wrong?

\subsection{A Correct Specification}

\subsection{A Formalized Specification}

\section{Conclusion}

\bibliography{imps}
\bibliographystyle{splncs04}

\setcounter{tocdepth}{1}
\listoftodos
\setcounter{tocdepth}{0}

\end{document}
