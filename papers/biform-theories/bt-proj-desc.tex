\documentclass[fleqn]{llncs}
\usepackage{latexsym}
\usepackage{amssymb,amsmath}
\usepackage{stmaryrd}
\usepackage{graphicx}
%\usepackage{color}
\usepackage{hyperref}
\usepackage{phonetic}
\usepackage{xargs}
\usepackage[pdftex,dvipsnames]{xcolor}
\usepackage{listings}

\lstset{language=ml,basicstyle=\ttfamily\small,breaklines=true,showspaces=false,
  showstringspaces=false,breakatwhitespace=true,texcl=true,
  escapeinside={(*}{*)}}

\input{bt-proj-desc-def}

\usepackage[colorinlistoftodos,textsize=tiny]{todonotes}
\newcommandx{\unsure}[2][1=]{\todo[linecolor=red,backgroundcolor=red!25,bordercolor=red,#1]{#2}}
\newcommandx{\change}[2][1=]{\todo[linecolor=blue,backgroundcolor=blue!25,bordercolor=blue,#1]{#2}}
\newcommandx{\info}[2][1=]{\todo[linecolor=OliveGreen,backgroundcolor=OliveGreen!25,bordercolor=OliveGreen,#1]{#2}}
\newcommandx{\improvement}[2][1=]{\todo[linecolor=Plum,backgroundcolor=Plum!25,bordercolor=Plum,#1]{#2}}

\title{Biform Theories: Project Description\thanks{This research was supported by NSERC.}}

\author{Jacques Carette, William M. Farmer, and Yasmine Sharoda}

\institute{%
Computing and Software, McMaster University, Canada\\
\url{http://www.cas.mcmaster.ca/~carette}\\
\url{http://imps.mcmaster.ca/wmfarmer}\\[1.5ex]
15 April 2018
}

\pagestyle{headings}

\begin{document}

\maketitle

\begin{abstract}
A \emph{biform theory} is a combination of an axiomatic theory and an
algorithmic theory that supports the integration of reasoning and
computation.  These are ideal for specifying and reasoning about
algorithms that manipulate mathematical expressions.  However,
formalizing biform theories is challenging since it requires the means
to express statements about the interplay of what these algorithms do
and what their actions mean mathematically.  This paper describes a
project to develop a methodology for expressing and managing
mathematical knowledge as a network of biform theories.  It is a
subproject of MathScheme, a long-term project at McMaster University
to produce a framework for integrating formal deduction and symbolic
computation.
\end{abstract}

\iffalse 

\textbf{Keywords:} Axiomatic mathematics, algorithmic mathematics, biform
theories, symbolic computation, reasoning about syntax, meaning
formulas, theory graphs.

\fi

\section{Problem}\label{sec:problem}

\section{Biform Theories}

\section{Project Objective}

\section{Work Plan}

\section{Related Work}

\section{Conclusion}

\bibliography{imps,hol-light-qe}
\bibliographystyle{plain}

\setcounter{tocdepth}{1}
\listoftodos
\setcounter{tocdepth}{0}

\end{document}
