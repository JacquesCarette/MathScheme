\section{${\bf CTT}_{\bf uqe}$ Formalization}\label{app:cttuqe}

\setcounter{biformthy}{0}

{\churchqe}~\cite{FarmerArxiv16} is a version of simple type theory
with quotation and evaluation.  {\churchuqe}~\cite{FarmerArxiv17} is a
variant of {\churchqe} that admits undefined expressions, partial
functions, and multiple base types of individuals.  {\churchuqe} also
includes a definite description operator and conditional expression
operator.  This appendix presents a formalization of the biform theory
graph test case in {\churchuqe} (instead of {\churchqe}) since
{\churchuqe} contains a notion of theory morphism.  The reader is
expected to be familiar with the notation of {\churchuqe}.  The
following are additional notes to the reader:

\be

  \item $\set{o,\epsilon,\iota}$ is the set of base types for all
    eight biform theories given below.  The single nonlogical base
    type $\iota$ is used to represent the natural numbers.

  \item All constants that are not introduced as components of one of
    the biform theories listed below are logical constants of
    {\churchuqe}, either primitive or defined.
    $\mname{is-abs}_{\epsilon \tarrow o}$, $\mname{abs-body}_{\epsilon
      \tarrow \epsilon}$, and $\mname{is-closed}_{\epsilon \tarrow o}$
    are defined logical constants not in~\cite{FarmerArxiv16}.
    $\mname{is-abs}_{\epsilon \tarrow o} \, \textbf{A}_\epsilon$ holds
    iff $\textbf{A}_\epsilon$ represents an abstraction.  If
    $\textbf{A}_\epsilon$ represents an abstraction, then
    $\mname{abs-body}_{\epsilon \tarrow \epsilon} \,
    \textbf{A}_\epsilon$ represents the body of the abstraction.
    $\mname{is-closed}_{\epsilon \tarrow o} \, \textbf{A}_\epsilon$
    holds iff $\textbf{A}_\epsilon$ represents an expression that is
    closed (and eval-free).

  \item The type attached to a constant may be dropped when there is
    no loss of meaning.

\iffalse
  \item When it makes sense, the notation
    $\set{\textbf{A}_{\alpha}^{1}, \ldots, \textbf{A}_{\alpha}^{n}}$
    denotes the predicate
    \[\LambdaApp \textbf{x}_\alpha \mdot 
    (\textbf{x}_\alpha = \textbf{A}_{\alpha}^{1} \OR \cdots \OR
    \textbf{x}_\alpha = \textbf{A}_{\alpha}^{n}).\]
\fi

  \item A constant of a type of the form $\epsilon \tarrow \cdots
    \tarrow \epsilon$ that is intended to be implemented by a
    transformer has a name in upper case letters.

  \item Expressions of type $\epsilon$, i.e., expressions that denote
    constructions, are colored red to identify where reasoning about
    syntax occurs.

\ee

The following are the eight biform theories and the two noninclusive
theory morphisms in the test case:

\begin{biformthy}[BT\thebiformthy: Simple Theory of $0$ and $S$]\em
\bi

  \item[] 

  \item[] \textbf{Primitive Base Types}

  \be

    \item $\iota$ (type of natural numbers).

  \ee

  \item[] \textbf{Primitive Constants}

  \be

    \item $0_\iota$.

    \item $S_{\iota \tarrow \iota}$.

  \ee

  \item[] \textbf{Defined Constants (selected)}

  \be

    \item $1_\iota = S \, 0_\iota$.

    \item $\mname{IS-FO-BT1}_{\epsilon \tarrow \epsilon} = \LambdaApp
      x_\epsilon \mdot \textbf{B}_\epsilon$ {\sglsp} where
      $\textbf{B}_\epsilon$ is a complex expression such that
      $\syn{(\LambdaApp x_\epsilon \mdot \textbf{B}_\epsilon) \,
        \synbrack{\textbf{A}_\alpha}}$ equals $\syn{\synbrack{T_o}}$
      [$\syn{\synbrack{F_o}}$] if $\textbf{A}_\alpha$ is [not] a term
      or formula of first-order logic with equality whose variables
      are of type $\iota$ and whose nonlogical constants are members
      of $\set{0_\iota,S_{\iota \tarrow \iota}}$.

  \ee

  \item[] \textbf{Axioms}

  \be

    \item $S \, x_\iota \not= 0_\iota$.

    \item $S \, x_\iota = S \, y_\iota \Implies x_\iota =
      y_\iota$.

  \ee

  \item[] \textbf{Transformers}

  \be

    \item $\pi_1$ for $\mname{IS-FO-BT1}_{\epsilon \tarrow \epsilon}$
      is an efficient program that accesses the data stored in the
      data structures that represent expressions.

    \item $\pi_2$ for $\mname{IS-FO-BT1}_{\epsilon \tarrow \epsilon}$
      uses the definition of $\mname{IS-FO-BT1}_{\epsilon \tarrow
        \epsilon}$.

  \ee

\ei
\end{biformthy}

\begin{biformthy}[BT\thebiformthy: Simple Theory of $0$, $S$, and $+$]\em
\bi

  \item[]

  \item[] \textbf{Extended Theories} 

  \be

    \item BT1.

  \ee


  \item[] \textbf{Primitive Constants}

  \be

    \setcounter{enumi}{2}

    \item $+_{\iota \tarrow \iota \tarrow \iota}$ (infix).

    \item $\mname{BPLUS}_{\epsilon \tarrow \epsilon \tarrow \epsilon}$ (infix).

  \ee

  \item[] \textbf{Defined Constants (selected)}

  \be

    \setcounter{enumi}{2}

    \item $\mname{bnat}_{\iota \tarrow \iota \tarrow \iota} =
      \LambdaApp x_\iota \mdot \LambdaApp y_\iota \mdot ((x_\iota +
      x_\iota) + y_\iota)$.
      
    Notational definition:

    \bi

      \item[] $(0)_2 = \mname{bnat} \, 0_\iota \, 0_\iota$.
  
      \item[] $(1)_2 = \mname{bnat} \, 0_\iota \, 1_\iota$.
  
      \item[] $(a_1 \cdots a_n0)_2 = \mname{bnat} \, (a_1 \cdots
        a_n)_2 \, 0_\iota$ {\sglsp} where each $a_i$ is 0 or 1.
  
      \item[] $(a_1 \cdots a_n1)_2 = \mname{bnat} \, (a_1 \cdots
        a_n)_2 \, 1_\iota$ {\sglsp} where each $a_i$ is 0 or 1.
  
    \ei

    \item $\mname{is-bnum}_{\epsilon \tarrow o} = 
      \IotaApp f_{\epsilon \tarrow o} \mdot
      \ForallApp \syn{u_\epsilon} \mdot
      (f_{\epsilon \tarrow \epsilon} \, \syn{u_\epsilon} \Iff {}\\
      \hspace*{2ex}\ForsomeApp \syn{v_\epsilon} \mdot 
      \ForsomeApp \syn{w_\epsilon} \mdot
      (\syn{u_\epsilon} = \syn{\synbrack{\mname{bnat} \, 
      \commabrack{v_\epsilon} \, \commabrack{w_\epsilon}}} \And {}\\
      \hspace*{4ex}(\syn{v_\epsilon} = \syn{\synbrack{0_\iota}} \OR 
      f_{\epsilon \tarrow \epsilon} \, \syn{v_\epsilon}) \And
      (\syn{w_\epsilon} = \syn{\synbrack{0_\iota}} \OR 
      \syn{w_\epsilon} = \syn{\synbrack{1_\iota}})))$.

    \item $\mname{IS-FO-BT2}_{\epsilon \tarrow \epsilon} = \LambdaApp
      x_\epsilon \mdot \textbf{B}_\epsilon$ {\sglsp} where
      $\textbf{B}_\epsilon$ is a complex expression such that
      $\syn{(\LambdaApp x_\epsilon \mdot \textbf{B}_\epsilon) \,
        \synbrack{\textbf{A}_\alpha}}$ equals $\syn{\synbrack{T_o}}$
      [$\syn{\synbrack{F_o}}$] if $\textbf{A}_\alpha$ is [not] a term
      or formula of first-order logic with equality whose variables
      are of type $\iota$ and whose nonlogical constants are members
      of $\set{0_\iota,S_{\iota \tarrow \iota},+_{\iota \tarrow \iota
          \tarrow \iota}}$.
      
  \ee

  \item[] \textbf{Axioms}

  \be

    \setcounter{enumi}{2}

    \item $x_\iota + 0_\iota = x_\iota$.

    \item $x_\iota + S \, y_\iota = S \, (x_\iota + y_\iota)$.

    \item $\mname{is-bnum} \, \syn{u_\epsilon} \Implies
      \syn{u_\epsilon \; \mname{BPLUS} \; \synbrack{(0)_2}} =
      \syn{u_\epsilon}$.

    \item $\mname{is-bnum} \, \syn{u_\epsilon} \Implies
      \syn{\synbrack{(0)_2} \; \mname{BPLUS} \; u_\epsilon} =
      \syn{u_\epsilon}$.

    \item $\syn{\synbrack{(1)_2} \; \mname{BPLUS} \;
      \synbrack{(1)_2}} = \syn{\synbrack{(10)_2}}$.

    \item $\mname{is-bnum} \, \syn{u_\epsilon} \Implies {}\\
        \hspace*{2ex} \syn{\synbrack{\mname{bnat} \,
            \commabrack{u_\epsilon} \, 0_\iota} \; \mname{BPLUS} \;
          \synbrack{(1)_2}} = \syn{\synbrack{\mname{bnat} \,
            \commabrack{u_\epsilon} \, 1_\iota}}$.

    \item $\mname{is-bnum} \, \syn{u_\epsilon} \Implies {}\\
        \hspace*{2ex} \syn{\synbrack{\mname{bnat} \,
            \commabrack{u_\epsilon} \, 1_\iota} \; \mname{BPLUS} \;
          \synbrack{(1)_2}} = \syn{\synbrack{\mname{bnat} \,
            \commabrack{u_\epsilon \; \mname{BPLUS} \;
              \synbrack{(1)_2}} \, 0_\iota}}$.

    \item $\mname{is-bnum} \, \syn{u_\epsilon} \Implies {}\\
        \hspace*{2ex} \syn{\synbrack{(1)_2} \; \mname{BPLUS} \;
          \synbrack{\mname{bnat} \, \commabrack{u_\epsilon} \, 0_\iota}} =
        \syn{\synbrack{\mname{bnat} \, \commabrack{u_\epsilon} \,
            1_\iota}}$.

    \item $\mname{is-bnum} \, \syn{u_\epsilon} \Implies {}\\
        \hspace*{2ex} \syn{\synbrack{(1)_2} \; \mname{BPLUS} \;
          \synbrack{\mname{bnat} \, \commabrack{u_\epsilon} \, 0_\iota}} =
        \syn{\synbrack{\mname{bnat} \, \commabrack{u_\epsilon \;
              \mname{BPLUS} \; \synbrack{(1)_2}} \, 0_\iota}}$.

    \item $(\mname{is-bnum} \, \syn{u_\epsilon} \And \mname{is-bnum}
      \, \syn{v_\epsilon}) \Implies {}\\
        \hspace*{2ex} \syn{\synbrack{\mname{bnat} \,
            \commabrack{u_\epsilon} \, 0_\iota} \; \mname{BPLUS} \;
          \synbrack{\mname{bnat} \, \commabrack{v_\epsilon} 0_\iota}} =
        \syn{\synbrack{\mname{bnat} \, \commabrack{u_\epsilon \;
              \mname{BPLUS} \; v_\epsilon} \, 0_\iota}}$.

    \item $(\mname{is-bnum} \, \syn{u_\epsilon} \And \mname{is-bnum}
      \, \syn{v_\epsilon}) \Implies {}\\
        \hspace*{2ex} \syn{\synbrack{\mname{bnat} \,
            \commabrack{u_\epsilon} \, 0_\iota} \; \mname{BPLUS} \;
          \synbrack{\mname{bnat} \, \commabrack{v_\epsilon} \, 1_\iota}} =
        \syn{\synbrack{\mname{bnat} \, \commabrack{u_\epsilon \;
              \mname{BPLUS} \; v_\epsilon} \, 1_\iota}}$.

    \item $(\mname{is-bnum} \, \syn{u_\epsilon} \And \mname{is-bnum}
      \, \syn{v_\epsilon}) \Implies {}\\
        \hspace*{2ex} \syn{\synbrack{\mname{bnat} \,
            \commabrack{u_\epsilon} \, 1_\iota} \; \mname{BPLUS} \;
          \synbrack{\mname{bnat} \, \commabrack{v_\epsilon} \, 0_\iota}} =
        \syn{\synbrack{\mname{bnat} \, \commabrack{u_\epsilon \;
              \mname{BPLUS} \; v_\epsilon} \, 1_\iota}}$.

    \item $(\mname{is-bnum} \, \syn{u_\epsilon} \And \mname{is-bnum}
      \, \syn{v_\epsilon}) \Implies {}\\
        \hspace*{2ex} \syn{\synbrack{\mname{bnat} \,
            \commabrack{u_\epsilon} \, 1_\iota} \; \mname{BPLUS} \;
          \synbrack{\mname{bnat} \, \commabrack{v_\epsilon} \, 1_\iota}} = {}\\
        \hspace*{2ex}\syn{\synbrack{\mname{bnat} \,
            \commabrack{(u_\epsilon \; \mname{BPLUS} \; v_\epsilon)
              \; \mname{BPLUS} \; \synbrack{(1)_2}} \, 0_\iota}}$.

  \ee

  \item[] \textbf{Transformers}

  \be

      \setcounter{enumi}{2}

    \item $\pi_3$ for $\mname{BPLUS}_{\epsilon \tarrow \epsilon
      \tarrow \epsilon}$ is an efficient program that satisfies Axioms
      5--15.

    \item $\pi_4$ for $\mname{BPLUS}_{\epsilon \tarrow \epsilon
      \tarrow \epsilon}$ uses Axioms 5--15 as conditional rewrite
      rules.

    \item $\pi_5$ for $\mname{IS-FO-BT2}_{\epsilon \tarrow \epsilon}$
      is an efficient program that accesses the data stored in the
      data structures that represent expressions.

    \item $\pi_6$ for $\mname{IS-FO-BT2}_{\epsilon \tarrow \epsilon}$
      uses the definition of $\mname{IS-FO-BT2}_{\epsilon \tarrow
        \epsilon}$.

  \ee
    
  \item[] \textbf{Theorems (selected)}

    \be

      \setcounter{enumi}{0}

      \item Meaning formula schema for 
      $\mname{BPLUS}_{\epsilon \tarrow \epsilon \tarrow \epsilon}$

      $((\mname{is-bnum} \, \syn{\textbf{A}_\epsilon} \And 
      \mname{is-bnum} \, \syn{\textbf{B}_\epsilon}) 
      \Implies {}\\
      \hspace*{2ex}(\mname{is-bnum} \, 
      \syn{(\textbf{A}_\epsilon \; \mname{BPLUS} \; \textbf{B}_\epsilon)} \And {}\\
      \hspace*{2.8ex}(\sembrack{\syn{\textbf{A}_\epsilon \; \mname{BPLUS} \; 
      \textbf{B}_\epsilon}}_\iota = 
      \sembrack{\syn{\textbf{A}_\epsilon}}_\iota + 
      \sembrack{\syn{\textbf{B}_\epsilon}}_\iota)))$.

    \ee

\ei
\end{biformthy}

\begin{biformthy}[BT\thebiformthy: Simple Theory of $0$, $S$, $+$, and $*$]\em

\bi

  \item[]

  \item[] \textbf{Extended Theories} 

  \be

    \setcounter{enumi}{1}

    \item BT2.

  \ee

  \item[] \textbf{Primitive Constants}

  \be

    \setcounter{enumi}{4}

    \item $*_{\iota \tarrow \iota \tarrow \iota}$ (infix).

    \item $\mname{BTIMES}_{\epsilon \tarrow \epsilon \tarrow \epsilon}$ (infix).

  \ee

  \item[] \textbf{Defined Constants (selected)}

  \be

    \setcounter{enumi}{3}

    \item $\mname{IS-FO-BT3}_{\epsilon \tarrow \epsilon} = \LambdaApp
      x_\epsilon \mdot \textbf{B}_\epsilon$ {\sglsp} where
      $\textbf{B}_\epsilon$ is a complex expression such that
      $\syn{(\LambdaApp x_\epsilon \mdot \textbf{B}_\epsilon) \,
        \synbrack{\textbf{A}_\alpha}}$ equals $\syn{\synbrack{T_o}}$
      [$\syn{\synbrack{F_o}}$] if $\textbf{A}_\alpha$ is [not] a term
      or formula of first-order logic with equality whose variables
      are of type $\iota$ and whose nonlogical constants are members
      of $\set{0_\iota,S_{\iota \tarrow \iota},+_{\iota \tarrow \iota
          \tarrow \iota},*_{\iota \tarrow \iota \tarrow \iota}}$.

  \ee

  \item[] \textbf{Axioms}

  \be

    \setcounter{enumi}{15}

    \item $x_\iota * 0_\iota = 0_\iota$.

    \item $x_\iota * S \, y_\iota = (x_\iota * y_\iota) + x_\iota$.

    \item $\mname{is-bnum} \, \syn{u_\epsilon} \Implies
      \syn{u_\epsilon \; \mname{BTIMES} \; \synbrack{(0)_2}} =
      \syn{\synbrack{(0)_2}}$.

    \item $\mname{is-bnum} \, \syn{u_\epsilon} \Implies \syn{
      \synbrack{(0)_2} \; \mname{BTIMES} \; u_\epsilon} =
      \syn{\synbrack{(0)_2}}$.

    \item $\mname{is-bnum} \, \syn{u_\epsilon} \Implies
      \syn{u_\epsilon \; \mname{BTIMES} \; \synbrack{(1)_2}} =
      \syn{u_\epsilon}$.

    \item $\mname{is-bnum} \, \syn{u_\epsilon} \Implies
      \syn{\synbrack{(1)_2} \; \mname{BTIMES} \; u_\epsilon} =
      \syn{u_\epsilon}$.

    \item $(\mname{is-bnum} \, \syn{u_\epsilon} \And \mname{is-bnum}
      \, \syn{v_\epsilon}) \Implies {}\\
        \hspace*{2ex} \syn{\synbrack{\mname{bnat} \,
            \commabrack{u_\epsilon} \, 0} \; \mname{BTIMES} \;
          \syn{v_\epsilon}} = \syn{\synbrack{\mname{bnat} \,
            \commabrack{u_\epsilon \; \mname{BTIMES} \; v_\epsilon} \,
            0_\iota}}$.

    \item $(\mname{is-bnum} \, \syn{u_\epsilon} \And \mname{is-bnum}
      \, \syn{v_\epsilon}) \Implies {}\\
        \hspace*{2ex} \syn{\synbrack{\mname{bnat} \,
            \commabrack{u_\epsilon} \, 1_\iota} \; \mname{BTIMES} \;
          \syn{v_\epsilon}} = \syn{\synbrack{\mname{bnat} \,
            \commabrack{u_\epsilon \; \mname{BTIMES} \; v_\epsilon} \,
            0_\iota} \; \mname{badd} \; v_\epsilon}$.

    \item $(\mname{is-bnum} \, \syn{u_\epsilon} \And \mname{is-bnum}
      \, \syn{v_\epsilon}) \Implies {}\\
        \hspace*{2ex} \syn{\syn{v_\epsilon}\; \mname{BTIMES} \;
          \synbrack{\mname{bnat} \, \commabrack{u_\epsilon} \, 0_\iota}} =
        \syn{\synbrack{\mname{bnat} \, \commabrack{u_\epsilon \;
              \mname{BTIMES} \; v_\epsilon} \, 0_\iota}}$.

    \item $(\mname{is-bnum} \, \syn{u_\epsilon} \And 
        \mname{is-bnum} \, \syn{v_\epsilon}) \Implies {}\\
        \hspace*{2ex} \syn{\syn{v_\epsilon} \; \mname{BTIMES} \;
          \synbrack{\mname{bnat} \, \commabrack{u_\epsilon} \, 1_\iota}} =
        \syn{\synbrack{\mname{bnat} \, \commabrack{u_\epsilon \;
              \mname{BTIMES} \; v_\epsilon} \, 0_\iota} \; \mname{badd} \;
          v_\epsilon}$.


  \ee

  \item[] \textbf{Transformers}

  \be

      \setcounter{enumi}{6}

    \item $\pi_7$ for $\mname{BTIMES}_{\epsilon \tarrow \epsilon
      \tarrow \epsilon}$ is an efficient program that satisfies Axioms
      18--25.

    \item $\pi_8$ for $\mname{BTIMES}_{\epsilon \tarrow \epsilon
      \tarrow \epsilon}$ uses Axioms 18--25 as conditional rewrite
      rules.

    \item $\pi_9$ for $\mname{IS-FO-BT3}_{\epsilon \tarrow \epsilon}$
      is an efficient program that accesses the data stored in the
      data structures that represent expressions.

    \item $\pi_{10}$ for $\mname{IS-FO-BT3}_{\epsilon \tarrow
      \epsilon}$ uses the definition of $\mname{IS-FO-BT3}_{\epsilon
      \tarrow \epsilon}$.

  \ee

  \item[] \textbf{Theorems (selected)}

  \be

    \setcounter{enumi}{1}

    \item Meaning formula schema for
    $\mname{BTIMES}_{\epsilon \tarrow \epsilon \tarrow \epsilon}$

    $((\mname{is-bnum} \, \syn{\textbf{A}_\epsilon} \And 
      \mname{is-bnum} \, \syn{\textbf{B}_\epsilon}) 
      \Implies {}\\
      \hspace*{2ex}(\mname{is-bnum} \, 
      \syn{(\textbf{A}_\epsilon \; \mname{BTIMES} \; \textbf{B}_\epsilon)} \And {}\\
      \hspace*{2.8ex}(\sembrack{\syn{\textbf{A}_\epsilon \; \mname{BTIMES} \; 
      \textbf{B}_\epsilon}}_\iota = 
      \sembrack{\syn{\textbf{A}_\epsilon}}_\iota * 
      \sembrack{\syn{\textbf{B}_\epsilon}}_\iota)))$.

  \ee

\ei
\end{biformthy}

\begin{biformthy}[BT\thebiformthy: Robinson Arithmetic (Q)]\em
\bi

  \item[]

  \item[] \textbf{Extended Theories} 

  \be

    \setcounter{enumi}{2}

    \item BT3.

  \ee

  \item[] \textbf{Axioms}

  \be

    \setcounter{enumi}{25}

    \item $x_\iota = 0_\iota \OR \ForsomeApp y_\iota \mdot S \,
      y_\iota = x_\iota$.

  \ee

\ei
\end{biformthy}

\begin{biformthy}[BT\thebiformthy: Complete Theory of $0$ and $S$]\em

\bi

  \item[]

  \item[] \textbf{Extended Theories} 

  \be

    \setcounter{enumi}{0}

    \item BT1.

  \ee

  \item[] \textbf{Primitive Constants}

  \be

    \setcounter{enumi}{6}

    \item $\mname{BT5-DEC-PROC}_{\epsilon \tarrow \epsilon}$.

  \ee

  \item[] \textbf{Defined Constants (selected)}

  \be

    \setcounter{enumi}{5}

    \item $\mname{IS-FO-BT1-ABS}_{\epsilon \tarrow \epsilon} = {}\\
    \hspace*{2ex}\LambdaApp \syn{x_\epsilon} \mdot 
    (\If \; (\mname{is-abs}_{\epsilon \tarrow o} \, \syn{x_\epsilon}) \;
    \syn{(\mname{IS-FO-BT1}_{\epsilon \tarrow \epsilon} \,
    (\mname{abs-body}_{\epsilon \tarrow \epsilon} \, x_\epsilon))} \;
    \syn{\synbrack{F_o}})$.

  \ee

  \item[] \textbf{Axioms}

  \be

    \setcounter{enumi}{26}

    \item Induction Schema for $S$

    $\ForallApp \syn{f_\epsilon} \mdot 
    ((\mname{is-expr}_{\epsilon \tarrow o}^{\iota \tarrow o} \, \syn{f_\epsilon} \And
    \sembrack{\syn{\mname{IS-FO-BT1-ABS}_{\epsilon \tarrow \epsilon} \, 
    f_\epsilon}}_o) \Implies {} \\
    \hspace*{2ex}((\sembrack{\syn{f_\epsilon}}_{\iota \tarrow o} \, 0_\iota \And
    (\ForallApp x_\iota \mdot 
    \sembrack{\syn{f_\epsilon}}_{\iota \tarrow o} \, x_\iota \Implies
    \sembrack{\syn{f_\epsilon}}_{\iota \tarrow o} \, 
    (\mname{S} \, x_\iota))) \Implies 
    \ForallApp x_\iota \mdot 
    \sembrack{\syn{f_\epsilon}}_{\iota \tarrow o} \, x_\iota))$.

    \item Meaning Formula for $\mname{BT5-DEC-PROC}_{\epsilon \tarrow \epsilon}$

    $\ForallApp \syn{u_\epsilon} \mdot 
     ((\mname{is-expr}_{\epsilon \tarrow o}^{o} \, 
     \syn{u_\epsilon} \And
     \mname{is-closed}_{\epsilon \tarrow o} \, \syn{u_\epsilon} \And 
     \sembrack{\syn{\mname{IS-FO-BT1}_{\epsilon \tarrow \epsilon} \, 
     u_\epsilon}}_o) \Implies {}\\
     \hspace*{2ex}((\syn{\mname{BT5-DEC-PROC}_{\epsilon \tarrow \epsilon} \, u_\epsilon} = 
     \syn{\synbrack{T_o}} \OR 
     \syn{\mname{BT5-DEC-PROC}_{\epsilon \tarrow \epsilon} \, u_\epsilon} = 
     \syn{\synbrack{F_o}}) \And {}\\
     \hspace*{3.1ex}
     \sembrack{\syn{\mname{BT5-DEC-PROC}_{\epsilon \tarrow \epsilon} \, u_\epsilon}}_o =
     \sembrack{\syn{u_\epsilon}}_o))$.

  \ee

  \item[] \textbf{Transformers}

  \be

    \setcounter{enumi}{10}

    \item $\pi_{11}$ for $\mname{BT5-DEC-PROC}_{\epsilon \tarrow
      \epsilon \tarrow \epsilon}$ is an efficient decision procedure
      that satisfies Axiom 28.


    \item $\pi_{12}$ for $\mname{IS-FO-BT1-ABS}_{\epsilon \tarrow
      \epsilon}$ is an efficient program that accesses the data stored
      in the data structures that represent expressions.

    \item $\pi_{13}$ for $\mname{IS-FO-BT1-ABS}_{\epsilon \tarrow
      \epsilon}$ uses the definition of
      $\mname{IS-FO-BT1-ABS}_{\epsilon \tarrow \epsilon}$.

  \ee

\ei
\end{biformthy}

\begin{biformthy}[BT\thebiformthy: Presburger Arithmetic]\em

\bi

  \item[]

  \item[] \textbf{Extended Theories} 

  \be

    \item[1.] BT2.

    \item[5.] BT5.

  \ee
    
  \item[] \textbf{Primitive Constants}

  \be

    \setcounter{enumi}{7}

    \item $\mname{BT6-DEC-PROC}_{\epsilon \tarrow \epsilon}$.

  \ee

  \item[] \textbf{Defined Constants (selected)}

  \be

    \setcounter{enumi}{6}

    \item $\mname{IS-FO-BT2-ABS}_{\epsilon \tarrow \epsilon} = {}\\
    \hspace*{2ex}\LambdaApp \syn{x_\epsilon} \mdot 
    (\If \; (\mname{is-abs}_{\epsilon \tarrow o} \, \syn{x_\epsilon}) \;
    \syn{(\mname{IS-FO-BT2}_{\epsilon \tarrow \epsilon} \,
    (\mname{abs-body}_{\epsilon \tarrow \epsilon} \, x_\epsilon))} \;
    \syn{\synbrack{F_o}})$.

  \ee

  \item[] \textbf{Axioms}

  \be

    \setcounter{enumi}{28}

    \item Induction Schema for $S$ and $+$

    $\ForallApp \syn{f_\epsilon} \mdot 
    ((\mname{is-expr}_{\epsilon \tarrow o}^{\iota \tarrow o} \, \syn{f_\epsilon} \And
    \sembrack{\syn{\mname{IS-FO-BT2-ABS}_{\epsilon \tarrow \epsilon} \, 
    f_\epsilon}}_o) \Implies {} \\
    \hspace*{2ex}((\sembrack{\syn{f_\epsilon}}_{\iota \tarrow o} \, 0_\iota \And
    (\ForallApp x_\iota \mdot 
    \sembrack{\syn{f_\epsilon}}_{\iota \tarrow o} \, x_\iota \Implies
    \sembrack{\syn{f_\epsilon}}_{\iota \tarrow o} \, 
    (\mname{S} \, x_\iota))) \Implies 
    \ForallApp x_\iota \mdot 
    \sembrack{\syn{f_\epsilon}}_{\iota \tarrow o} \, x_\iota))$.

    \item Meaning formula for $\mname{BT6-DEC-PROC}_{\epsilon \tarrow \epsilon}$.


    $\ForallApp \syn{u_\epsilon} \mdot 
    ((\mname{is-expr}_{\epsilon \tarrow o}^{o} \, 
    \syn{u_\epsilon}  \And
    \mname{is-closed}_{\epsilon \tarrow o} \, \syn{u_\epsilon} \And 
    \sembrack{\syn{\mname{IS-FO-BT2}_{\epsilon \tarrow \epsilon} \, 
    u_\epsilon}}_o) \Implies {}\\
    \hspace*{2ex}((\syn{\mname{BT6-DEC-PROC}_{\epsilon \tarrow \epsilon} \, u_\epsilon} = 
    \syn{\synbrack{T_o}} \OR 
    \syn{\mname{BT6-DEC-PROC}_{\epsilon \tarrow \epsilon} \, u_\epsilon} = 
    \syn{\synbrack{F_o}}) \And {}\\
    \hspace*{3.1ex}
    \sembrack{\syn{\mname{BT6-DEC-PROC}_{\epsilon \tarrow \epsilon} \, u_\epsilon}}_o =
    \sembrack{\syn{u_\epsilon}}_o))$.

  \ee

  \item[] \textbf{Transformers}

  \be

    \setcounter{enumi}{13}

    \item $\pi_{14}$ for $\mname{BT6-DEC-PROC}_{\epsilon \tarrow
      \epsilon \tarrow \epsilon}$ is an efficient decision procedure
      that satisfies Axiom 30.


    \item $\pi_{15}$ for $\mname{IS-FO-BT2-ABS}_{\epsilon \tarrow
      \epsilon}$ is an efficient program that accesses the data
      stored in the data structures that represent expressions.

    \item $\pi_{16}$ for $\mname{IS-FO-BT2-ABS}_{\epsilon \tarrow
      \epsilon}$ uses the definition of
      $\mname{IS-FO-BT2-ABS}_{\epsilon \tarrow \epsilon}$.

  \ee
    
  \item[] \textbf{Theorems (selected)}

    \be

      \setcounter{enumi}{2}

      \item Meaning formula for 
      $\mname{BPLUS}_{\epsilon \tarrow \epsilon \tarrow \epsilon}$

      $\ForallApp \syn{u_\epsilon} \mdot \ForallApp \syn{v_\epsilon} \mdot
      ((\mname{is-bnum} \, \syn{u_\epsilon} \And \mname{is-bnum} \, \syn{v_\epsilon}) 
      \Implies {}\\
      \hspace*{2ex}(\mname{is-bnum} \, 
      \syn{(u_\epsilon \; \mname{BPLUS} \; v_\epsilon)} \And {}\\
      \hspace*{2.8ex}(\sembrack{\syn{u_\epsilon \; \mname{BPLUS} \; 
      v_\epsilon}}_\iota = 
      \sembrack{\syn{u_\epsilon}}_\iota + \sembrack{\syn{v_\epsilon}}_\iota)))$.

    \ee

\ei
\end{biformthy}

\begin{biformthy}[BT\thebiformthy: First-Order Peano Arithmetic]\em

\bi

  \item[]

  \item[] \textbf{Extended Theories} 

  \be

    \item[3.] BT3.

    \item[6.] BT6.

  \ee

  \item[] \textbf{Defined Constants (selected)}

  \be

    \setcounter{enumi}{7}

    \item $\mname{IS-FO-BT3-ABS}_{\epsilon \tarrow \epsilon} = {}\\
    \hspace*{2ex}\LambdaApp \syn{x_\epsilon} \mdot 
    (\If \; (\mname{is-abs}_{\epsilon \tarrow o} \, \syn{x_\epsilon}) \;
    \syn{(\mname{IS-FO-BT3}_{\epsilon \tarrow \epsilon} \,
    (\mname{abs-body}_{\epsilon \tarrow \epsilon} \, x_\epsilon))} \;
    \syn{\synbrack{F_o}})$.

  \ee

  \item[] \textbf{Axioms}

  \be

    \setcounter{enumi}{30}

    \item Induction Schema for $S$, $+$, and $*$

    $\ForallApp \syn{f_\epsilon} \mdot 
    ((\mname{is-expr}_{\epsilon \tarrow o}^{\iota \tarrow o} \, \syn{f_\epsilon} \And
    \sembrack{\syn{\mname{IS-FO-BT3-ABS}_{\epsilon \tarrow \epsilon} \, 
    f_\epsilon}}_o) \Implies {} \\
    \hspace*{2ex}((\sembrack{\syn{f_\epsilon}}_{\iota \tarrow o} \, 0_\iota \And
    (\ForallApp x_\iota \mdot 
    \sembrack{\syn{f_\epsilon}}_{\iota \tarrow o} \, x_\iota \Implies
    \sembrack{\syn{f_\epsilon}}_{\iota \tarrow o} \, 
    (\mname{S} \, x_\iota))) \Implies 
    \ForallApp x_\iota \mdot 
    \sembrack{\syn{f_\epsilon}}_{\iota \tarrow o} \, x_\iota))$.

  \ee

  \item[] \textbf{Transformers}

  \be

    \setcounter{enumi}{16}

    \item $\pi_{17}$ for $\mname{IS-FO-BT3-ABS}_{\epsilon \tarrow
      \epsilon}$ is an efficient program that accesses the data stored
      in the data structures that represent expressions.

    \item $\pi_{18}$ for $\mname{IS-FO-BT3-ABS}_{\epsilon \tarrow
      \epsilon}$ uses the definition of
      $\mname{IS-FO-BT3-ABS}_{\epsilon \tarrow \epsilon}$.

  \ee

  \item[] \textbf{Theorems (selected)}

  \be

    \setcounter{enumi}{3}

    \item Axiom 26.

    \item Meaning formula
    $\mname{BTIMES}_{\epsilon \tarrow \epsilon \tarrow \epsilon}$

    $\ForallApp \syn{u_\epsilon} \mdot \ForallApp \syn{v_\epsilon} \mdot
    ((\mname{is-bnum} \, \syn{u_\epsilon} \And \mname{is-bnum} \, \syn{v_\epsilon}) 
    \Implies {}\\
    \hspace*{2ex}(\mname{is-bnum} \, 
    \syn{(u_\epsilon \; \mname{BTIMES} \; v_\epsilon)} \And {}\\
    \hspace*{2.8ex}(\sembrack{\syn{u_\epsilon \; \mname{BTIMES} \; 
    v_\epsilon}}_\iota = 
    \sembrack{\syn{u_\epsilon}}_\iota * \sembrack{\syn{v_\epsilon}}_\iota)))$.

  \ee

\ei
\end{biformthy}

\begin{biformthy}[BT\thebiformthy: Higher-Order Peano Arithmetic]\em

\bi

  \item[]

  \item[] \textbf{Extended Theories} 

  \be

    \setcounter{enumi}{0}

    \item BT1.

  \ee

  \item[] \textbf{Defined Constants (selected)}

  \be

    \setcounter{enumi}{8}

    \item $+_{\iota \tarrow \iota \tarrow \iota} =
    \IotaApp f_{\iota \tarrow \iota \tarrow \iota} \mdot
    \ForallApp x_\iota \mdot \ForallApp y_\iota \mdot {}\\
    \hspace*{2ex} (f_{\iota \tarrow \iota \tarrow \iota} \, 
    x_\iota \, 0_\iota = x_\iota \And {}\\
    \hspace*{2.8ex}f_{\iota \tarrow \iota \tarrow \iota} \, 
    x_\iota \, (S \, y_\iota) = S \, 
    (f_{\iota \tarrow \iota \tarrow \iota} \, x_\iota \,
    y_\iota))$.

    \item $*_{\iota \tarrow \iota \tarrow \iota} =
    \IotaApp f_{\iota \tarrow \iota \tarrow \iota} \mdot
    \ForallApp x_\iota \mdot \ForallApp y_\iota \mdot {}\\
    \hspace*{2ex} (f_{\iota \tarrow \iota \tarrow \iota} \, 
    x_\iota \, 0_\iota = 0_\iota \And {}\\
    \hspace*{2.8ex}f_{\iota \tarrow \iota \tarrow \iota} \, 
    x_\iota \, (S \, y_\iota) =  
    (f_{\iota \tarrow \iota \tarrow \iota} \, x_\iota \,
    y_\iota) + x_\iota)$.

  \ee



  \item[] \textbf{Axioms}

  \be

    \setcounter{enumi}{31}

    \item Induction Axiom for the Natural Numbers

    $\ForallApp p_{\iota \tarrow o} \mdot ((p_{\iota \tarrow o} \, 0_\iota \And 
    (\ForallApp x_\iota \mdot (p_{\iota \tarrow o} \, x_\iota \Implies 
    p_{\iota \tarrow o} \, (S \, x_\iota)))) \Implies {}\\
    \hspace*{2ex}\ForallApp x_\iota \mdot 
    p_{\iota \tarrow o} \, x_\iota)$.

  \ee

  \item[] \textbf{Theorems (selected)}

  \be

    \setcounter{enumi}{5}

    \item Axiom 27 (Induction Schema for $S$).

    \item Axiom 39 (Induction Schema for $S$ and $+$).

    \item Axiom 31 (Induction Schema for $S$, $+$, and $*$).

  \ee

\ei
\end{biformthy}

\begin{thymorphism}[BT4-to-BT7]\em
\be

  \item[]

  \item[] \textbf{Source Theory} BT4.

  \item[] \textbf{Target Theory} BT7.

  \item[] \textbf{Translation} 

  \bi

    \item[] $\mu$ is defined as follows:

    \bi

      \item[] $\mu(o) = \LambdaApp x_o \mdot T_o$.

      \item[] $\mu(\epsilon) = \LambdaApp x_\epsilon \mdot T_o$.

      \item[] $\mu(\iota) = \LambdaApp x_\iota \mdot T_o$.

    \ei

    $\nu$ is the identity on the constants of BT4.

  \ei

  \item[] \textbf{Nontrivial Obligations} 

  \bi

    \item[] Axiom 26.

  \ei  

\ee
\end{thymorphism}

\begin{thymorphism}[BT7-to-BT8]\em
\be

  \item[]

  \item[] \textbf{Source Theory} BT7.

  \item[] \textbf{Target Theory} BT8.

  \item[] \textbf{Translation} 

  \bi

    \item[] $\mu$ is defined as follows:

    \bi

      \item[] $\mu(o) = \LambdaApp x_o \mdot T_o$.

      \item[] $\mu(\epsilon) = \LambdaApp x_\epsilon \mdot T_o$.

      \item[] $\mu(\iota) = \LambdaApp x_\iota \mdot T_o$.

    \ei

    $\nu$ is the identity on the constants of BT7.

  \ei

  \item[] \textbf{Nontrivial Obligations} 

  \bi

    \item[] Axioms 3--25, 27--31.

  \ei  

\ee
\end{thymorphism}
