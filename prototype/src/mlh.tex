\documentclass[11pt]{article}

\usepackage{alltt}

\setlength {\textheight} {7.5in}
\setlength {\textwidth} {11in}

\setlength{\parindent}{0pt}
\begin{document}

% A single space used by the OCaml formatter
\newcommand{\s}{\hspace*{1ex}}

% The following commands may seem very ambiguous due to their short names, yet
% there is a logical reasoning behind this choice. As one can read in configuration.ml
% file, OCaml's Format module works by counting characters to determine line breaks.
% In retrospect, OCaml cannot understand LaTeX commands, so it treats them as normal
% strings. Hence we try to name these commands relative to their actual "size" in the
% final document. This way OCaml's Format module can use space sparingly.
\newcommand{\lm}{\lambda}

% Used for operator names, i.e. "if", "and", "or", "E", ect.
\newcommand{\N}[1]{\mbox{\sf #1}}
\newcommand{\K}[1]{\mathrel{#1}}

\newcommand{\ab}[1]{\begin{alltt} \N{#1} \end{alltt}}

\newcommand{\F}{\forall\s}
\newcommand{\E}{\exists\s}
\newcommand{\U}{\exists!\s}
\newcommand{\I}{\iota}
\newcommand{\e}{\epsilon}

\newcommand{\Bl}{\lceil}
\newcommand{\Br}{\rceil}

\newcommand{\RA}{\rightarrow}
\newcommand{\LRA}{\Leftrightarrow}
\newcommand{\BR}{\mapsto}

\newcommand{\ty}{\N{type}}
\newcommand{\ts}{\N{type}\s}

\newcommand{\Thy}{\N{Theory}\s}
\newcommand{\alg}{\s\N{along}\s}
\newcommand{\extend}{\N{extended by}\s}
\newcommand{\extendCons}{\N{extended conservatively by}\s}
\newcommand{\comb}{\N{combine}\s}
\newcommand{\Con}{\N{Concepts}\s}
\newcommand{\Def}{\N{Definitions}\s}
\newcommand{\Var}{\N{Variables}\s}
\newcommand{\Fac}{\N{Facts}\s}
\newcommand{\Ind}{\N{Inductive}\s}

\newcommand{\ax}{\N{axiom}\s}
